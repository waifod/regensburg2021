\documentclass[a4paper,11pt,openany]{scrartcl}
\usepackage{../../basic}

\begin{document}

\title{Introduction to Stable Homotopy Theory\\ Assignment 1}

\author{Matteo Durante}

\maketitle

\exercise{1}
\begin{proof}
    Let us consider the continuous maps $|s_n|$, $|s_{n+1}|$. We can construct
    an homotopy $H$ between them by setting
    \begin{align*}
        H\colon|\Delta^n|\times|\Delta^1|&\rightarrow |\Delta^{n+1}| \\
        (t,\lambda) &\mapsto (1-\lambda)\cdot |s_n|(t)+\lambda\cdot
        |s_{n+1}|(t)
    \end{align*}

    Given a topological space $Y$ and the associated simplicial set $X=Sing(Y)$,
    consider two $n$-simplices $\sigma,\tau\in X_n$. If they are
    homotopic we have a $(n+1)$-simplex $\psi\in X_{n+1}$ such that
    $\partial_n\psi=\sigma$, $\partial_{n+1}\psi=\tau$ and
    $\partial_i\psi=s_i\partial_{n-1}\sigma=s_i\partial_{n-1}\tau$ for every $i$ such
    that $0\leq i\leq n-1$.

    As we know, $\psi$ is a continuous map $|\Delta^{n+1}|\rightarrow Y$ and as
    such we can compose it with $H$ to get a map $\psi\cdot
    H\colon|\Delta^n|\times|\Delta^1|\rightarrow Y$. By construction,
    $H|_{|\Delta^n|\times\{0\}}=|\partial_n|$ and
    $H|_{|\Delta^n|\times\{1\}}=|\partial_{n+1}|$, thus we have that $\psi\cdot
    H|_{|\Delta^n|\times\{0\}}=\sigma$, $\psi\cdot
    H|_{|\Delta^n|\times\{1\}}=\tau$. Also, $\psi\cdot
    H|_{|\partial\Delta^n|\times\{\lambda\}}$ satisfies the desired equalities.

    On the other hand, suppose that there is a homotopy $H\colon\Delta^n\times
    I\rightarrow Y$ such that $H(-,0)=\sigma$, $H(-,1)=\tau$,
    $H|_{\partial\Delta^n\times\{\lambda\}}=\sigma|_{\partial\Delta^n}=\tau|_{\partial\Delta^n}$
    for all $\lambda\in I$.
    We want to construct a $(n+1)$-simplex $\psi$ showing $\sigma,\tau$ to be
    homotopic. Since this is an equivalence relation, we shall prove that there
    is a $n$-simplex $\theta$ which is homotopic to both of them, which will
    conclude the proof. In order to do this, we only need to cover
    $\Delta^n\times\Delta^1$ by two $(n+1)$-simplices and then compose our covering with
    $H\colon\Delta^n\times\Delta^1\rightarrow X$. In order to do this, we
    will only specify where the verteces because, seeing these simplicial sets
    as categories (which does not lose any information since they lie in the
    image of the nerve functor), they correspond to an order and a preorder
    respectively.
    \begin{align*}
        t_0\colon\Delta^{n+1}&\rightarrow\Delta^n\times\Delta^1 \\
        v_i &\mapsto\begin{cases}
                        (v_i,0)\text{ if $i<n+1$,} \\
                        (v_{n+1},1)\text{ otherwise}
                    \end{cases} \\
        t_1\colon\Delta^{n+1}&\rightarrow\Delta^n\times\Delta^1 \\
        v_i &\mapsto\begin{cases}
                        (v_0,0)\text{ if $i=0$,} \\
                        (v_{i-1},1)\text{ otherwise}
                    \end{cases}
    \end{align*}
    Notice that the two simplices share a face, respectively their first ones.
    After proving that the homotopy relation is equivalent to the one we get by
    considering the $0$th and the first face, this gives us the proof.
\end{proof}

~\\
\exercise{2}
\begin{proof}
    Consider homotopies $H\colon X\times\Delta^1\rightarrow Y$, $H'\colon
    Y\times\Delta^1\rightarrow Z$, respectively between $f,f'$ and $g,g'$.
    Considering the maps $gH\colon X\times\Delta^1\rightarrow Z$,
    $H'(f'\times\id_{\Delta_1})\colon X\times\Delta^1\rightarrow Z$, we see that
    they define homotopies between $gf,gf'$ and $gf',g'f'$ respectively since
    $gH(-,0)=gf$, $gH(-,1)=gf'$ and
    $H'(f'\times\Delta^1)(-,0)=H'(f',0)=(H'(-,0))f'=gf'$,
    $H'(f'\times\Delta^1)(-,1)=H'(f',1)=(H'(-,1))f'=g'f'$.

    We now want to prove that the homotopy relation is transitive, which will
    then imply that $gf$ is homotopic to $g'f'$.

    Let us consider three morphisms of Kan complexes $f,g,h\colon X\rightarrow
    Y$ with homotopies $H\colon X\times\Delta^1\rightarrow Y$ between $f,g$ and
    $H'\colon X\times\Delta^1\rightarrow Y$ between $g,h$. We want to construct a morphism
    $H''\colon X\times\Lambda^2_1\rightarrow Y$ such that $H''(-,0)=f$,
    $H''(-,1)=g$ and $H''(-,2)=h$, extend it to a new one $H'''\colon
    X\times\Delta^2\rightarrow Y$ and then restrict it to
    $\tilde{H}=H'''(\id_X\times\partial^2_1)\colon X\times\Delta^1\rightarrow Y$.

    First of all, by adjunction we get from $H,H'$ two morphisms of simplicial
    sets $G,G'\colon\Delta^1\rightarrow\underline{\sSet}(X,Y)$, which we then glue together by
    considering the pushout diagram
    \[
        \begin{tikzcd}
            \Delta_0\ar[r, "\partial_0^1"]\ar[d, "\partial_1^1"]
            & \Delta_1\ar[d]\ar[ddr, bend left, "G"] \\
            \Delta_1\ar[r]\ar[drr, bend right, "G'"]
            & \Lambda_1^2\ar[dr, dotted, "\tilde{G}"] \\
            && \underline{\sSet}(X,Y)
        \end{tikzcd}
    \]
    Since $\sSeten(X,Y)$ is a Kan complex, we can fill the horn and get a morphism
    $\Delta^2\rightarrow\sSeten(X,Y)$, which by adjunction corresponds to a
    morphism $H'''\colon X\times\Delta^2\rightarrow Y$ such that
    $H'''(\id_X\times\partial^2_2)=H$, $H'''(\id_X\times\partial^2_0)=H'$.
    Composing this morphism with
    $\id_X\times\partial^2_1$, we get then the desired homotopy $\tilde{H}\colon
    X\times\Delta^1\rightarrow Y$ between $f$ and $h$.
\end{proof}

~\\
\exercise{3}
\begin{proof}
    Consider two Kan complexes $X,Y$, two morphisms $f,g\colon X\rightarrow Y$
    and an homotopy $H\colon X\times\Delta^1\rightarrow Y$ between them. We
    will abuse the notation and
    identify an $n$-simplex with the unique map from $\Delta^n$ identified by
    Yoneda.

    Given $x\in X_0$ and $t\in X_n$ such that $t|_{\partial\Delta^n}=x$ (here we
    mean that the map $\partial\Delta^n\rightarrow X$ given by composing
    $t\colon\Delta^n\rightarrow X$ with the inclusion
    $\partial\Delta^n\rightarrow\Delta^n$ maps all simplices to the degenerate
    ones identified by $x$), consider $ft,gt\in Y_n$,
    $\gamma_x=H|_{\{x\}\times\Delta^1}\colon\Delta^1\xrightarrow{(x,\id_{\Delta^1})}X\times\Delta^1\xrightarrow{H} Y$.
    By definition,
    $(ev_0)^{-1}_*\colon\pi_n(Y,fx)\rightarrow\pi_n(\sSeten(\Delta^1,Y),\gamma_x)$
    sends the class of $ft$ to the class an $n$-simplex
    $\alpha\colon\Delta^n\rightarrow\sSeten(\Delta^1,Y)$ such that
    $\alpha|_{\partial\Delta^n}=\gamma_x$ (again, all simplices are mapped to
    the degenerate ones simplex given bythe $0$-simplex $\gamma_x$) and
    corresponding to a morphism $\alpha'\colon\Delta^n\times\Delta^1\rightarrow
    Y$ satisfying $\alpha'|_{\Delta^n\times\{0\}}=ft$.

    We pick $\beta'\colon \Delta^n\times\Delta^1\xrightarrow{t\times\id_{\Delta^1}}X\times\Delta^1\xrightarrow{H}Y$
    as our $\alpha'$ since by construction
    $\beta'|_{\Delta^n\times\{0\}}=H(t,0)=ft$ and
    $\beta'|_{\partial\Delta^n\times\Delta^1}=H(t|_{\partial\Delta^n}\times\id_{\Delta_1})=H(x\times\Delta^1)=H|_{\{x\}\times\Delta^1}=\gamma_1$.
    $\beta$ is then sent by $(ev_1)_*$ to $\beta'|_{\Delta^n\times\{1\}}=gt$,
    which proves the commutativity of the diagram.

    Since this holds for every $n\geq 0$ and all $x\in X_0$ and the evaluation
    maps are isomorphisms, we have shown that
    an homotopy equivalence $H\colon X\times\Delta^1\rightarrow Y$ induces an
    isomorphism $(\gamma_x)_*\colon\pi_n(Y,fx)\rightarrow\pi_n(Y,gx)$ on
    homotopy groups.
\end{proof}
\end{document}

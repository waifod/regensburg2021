\documentclass[a4paper,11pt]{amsart}
\usepackage{amsmath,amssymb,bbm}
\usepackage[utf8]{inputenc}
\usepackage{stmaryrd} % Eckige Klammern [[ \llbracket \rrbracket ]]
\usepackage{graphicx}
\usepackage{hyperref}
\usepackage{xcolor}
\usepackage{mathrsfs}
\usepackage{mathtools}
\usepackage{tikz-cd}
\DeclareMathOperator{\Hom}{Hom}
\DeclareMathOperator{\Alg}{-Alg}
\DeclareMathOperator{\CAlg}{-CAlg}
\DeclareMathOperator{\R}{\mathbb{R}}
\DeclareMathOperator{\h}{\mathbb{H}}
\DeclareMathOperator{\Q}{\mathbb{Q}}
\DeclareMathOperator{\Z}{\mathbb{Z}}
\DeclareMathOperator{\N}{\mathbb{N}}
\DeclareMathOperator{\PP}{\mathcal{P}}
\DeclareMathOperator{\SP}{\mathcal{SP}}
\DeclareMathOperator{\id}{id}
\DeclareMathOperator{\GL}{GL}
\DeclareMathOperator{\re}{Re}
\DeclareMathOperator{\im}{Im}
\DeclareMathOperator{\rk}{rk}
\DeclareMathOperator{\mult}{mult}
\DeclareMathOperator{\proj}{proj}
\DeclareMathOperator{\Log}{Log}
\DeclarePairedDelimiter{\ceil}{\lceil}{\rceil}
\DeclarePairedDelimiter\floor{\lfloor}{\rfloor}
\usepackage{enumitem}
\usepackage{tabularx}
\thispagestyle{empty}


\voffset       =  -1in         
\hoffset       =  -1in
\textheight    = 250mm         
\textwidth     = 150mm %166 180mm x 260mm
\evensidemargin=  30mm %2cm Randlinks/rechts
\oddsidemargin =  30mm % 
\topmargin     =  12mm % 2cm Rand oben/unten
\headheight     =  12pt
\headsep        =   6mm
 
\def\stretch#1{\renewcommand{
\baselinestretch}{#1}\small\normalsize}
\stretch{1} % 1.5 zeilig !!!
\parindent = 0pt                         
\renewcommand{\labelenumi}
{\textit{\roman{enumi}})}
\parskip = \medskipamount  

\setenumerate{label=\alph*)}
\setenumerate[2]{label=\roman*)}

\newcommand\punkte[1]{\ifhmode\unskip\fi\hspace*{1em plus 1fill}\parbox[t]{1cm}{#1}}

%diagonal arrow
\makeatletter
\newcommand{\fixed@sra}{$\vrule height 2\fontdimen22\textfont2 width 0pt\shortrightarrow$}
\newcommand{\shortarrow}[1]{%
  \mathrel{\text{\rotatebox[origin=c]{\numexpr#1*45}{\fixed@sra}}}
}
\makeatother

\newcommand{\F}{\mathcal F}
\newcommand{\set}{\mathrm{Set}}
\newcommand{\sst}{\subseteq}
\newcommand{\ito}{\hookrightarrow}
\newcommand{\C}{\mathcal C}
\newcommand{\D}{\mathcal D}
\renewcommand{\hom}{\operatorname{Hom}}
\newcommand{\ilim}{\varinjlim}
\newcommand{\plim}{\varprojlim}
\newcommand{\op}{\mathrm{op}}
\newcommand{\e}{\varepsilon}
\newcommand{\wh}{\widehat}

%%%sheet3
\newcommand{\cat}{\mathrm{Cat}}
\newcommand{\gpd}{\mathrm{Gpd}}
\newcommand{\G}{\mathcal G}
\newcommand{\ob}{\mathrm{Ob}}
\newcommand{\disc}{\operatorname{disc}}
\newcommand{\wt}{\widetilde}


\begin{document}
\textbf{Name}\hfill\textbf{Semester}

\vspace*{-.1cm} Yuhao ZHANG\hfill WS 2020/21


\vspace*{.6cm}
\rule{\linewidth}{1pt}
\begin{center}
\Large
{\bf Higher Category Theory} \\
Solutions to Sheet 3
 % Vorlesungstitel
\end{center}
\rule{\linewidth}{1pt}
\\

%%%%%%%%%%%%%%%%%%%%%%%%%%%%%%%%%%%%%%%%%%%%%%%%%%%%%%%%%%%%%%%%%%%%%%%%%%%%%%%




%%%%%%%%%%%%%%%%%%%%%%%%%%%%%%%%%%%%%%%%%%%%%%%%%%%%%%%%%%%%%%%%%%%%%%%%%%%%%%%%


\textbf{Solution to Exercise 3.} \textit{Proof}. \textbf{(i)} It suffices to show that the functor $\Hom_\cat(-,\C)$ is represented by $\C^\simeq$ for each $\C\in\ob(\cat)$. To this end, we note that for every $\G\in\gpd$, any functor $F\colon\G\to\C$ factorizes uniquely through $\C^\simeq$, because $F(f)$ is an isomorphism for any (iso-)morphism $f$ in $\G$, and if $F$ factorizes as 
\[
\G\overset{F'}{\to}\C\ito\C^\simeq\text{ and }\G\overset{F''}{\to}\C\ito\C^\simeq
\]
then $F'=F''$ on objects while for any morphism $f$ in $\G$, $F'(f)=F(f)=F''(f)$ (so $F'=F''$). This gives a bijection
\[
\hom_\cat(\G,\C)\cong\hom_\gpd(\G,\C^\simeq).
\]
To see the functoriality, take any $G\colon\G\to\G'$ in $\gpd$. Then we have a commutative diagram
\[
\begin{tikzcd}
\hom_\cat(\G',\C)\arrow[r,equal,"\sim"]\arrow[d]& \hom_\gpd(\G',\C^\simeq)\arrow[d]\\
\hom_\cat(\G,\C)\arrow[r,equal,"\sim"]& \hom_\gpd(\G,\C^\simeq)
\end{tikzcd}
\ 
\begin{tikzcd}[column sep=small]
F\arrow[rr,mapsto]\arrow[d,mapsto]&& F'\arrow[d,mapsto]\\
F\circ G\arrow[r,mapsto]& (F\circ G)'\arrow[r,equal]& F'\circ G
\end{tikzcd}
\]
where $F$, $F\circ G$ factorize through $F'$, $(F\circ G)'$ respectively. Note that $F'\circ G=(F\circ G)'$ since the composite $\G\overset{\G}{\to}\G'\overset{F'}{\to}\C^\simeq\ito\C$ is $F\circ G$.

\textbf{(ii)} We claim that subgroupoids of $EX$ are of the form
\[
\coprod_{i\in I}EX_i
\]
where $(X_i)_{i\in I}$ is a family of disjoint subsets of $X$. Indeed, such subcateories $\coprod_{i\in I}EX_i$ is a groupoid, and thus a subgroupoid of $X$. On the other hand, for any subgroupoid $Y$ of $X$, we define $I$ to be the set of isomorphism classes of objects in $Y$. Therefore $Y=\coprod_{i\in I}Ei$, which can be seen from the fact that $\ob(Y)=\ob(\coprod_{i\in I}Ei)$ and for any $x,y\in\ob(Y)$, 
\[
\hom_Y(x,y)=\hom_{\coprod_IEi}(x,y)=\left\{\begin{array}{ll}
\varnothing& \text{ if $x,y$ are not isomorphic}\\
\{(x,y)\}&\text{ if $x,y$ are isomorphic}
\end{array}\right.
\]

\textbf{(iii)} It is enough to show that for all small set $X$, the functor $\hom_\set(\ob(-),X)$ is represented by $EX$. To this end, for any map $F\colon\ob(\C)\to X$, we define a functor $\wt F$ by letting
\begin{itemize}
\item $\wt F(x)=F(x)$ for any $x\in\ob(\C)$;

\item $\hom_\C(x,y)\to\hom_{EX}(Fx,Fy)$ is the constant map, sending each morphism $f\colon x\to y$ to $(Fx,Fy)$.
\end{itemize}
and we get a bijection
\begin{align*}
\hom_\set(\ob(\C),X)&\to\hom_\cat(\C,EX)\\
F&\mapsto \wt F\\
\ob(F)&\mapsfrom F
\end{align*}
(the verification of them being mutually inverse is straightforward).

As for the functoriality, take any functor $G\colon\C\to\C'$. Then the diagram
\[
\begin{tikzcd}
\hom_\set(\ob(\C'),X)\arrow[r,equal,"\sim"]\arrow[d]& \hom_\cat(\C',EX)\arrow[d]\\
\hom_\set(\ob(\C),X)\arrow[r,equal,"\sim"]& \hom_\cat(\C,EX)
\end{tikzcd}
\ 
\begin{tikzcd}[column sep=small]
F\arrow[rr,mapsto]\arrow[d,mapsto]&& \wt F\arrow[d,mapsto]\\
F\circ\ob(G)\arrow[r,mapsto]& \wt F\circ G\arrow[r,equal]& \wt{F\circ\ob(G)}
\end{tikzcd}
\]
is commutative. Here $\wt F\circ G=\wt{F\circ\ob(G)}$ because they both equal to $F\circ \ob(G)$ on objects and hence they are the same on morphisms (since the map between hom sets $\hom_\C(x,y)\to\hom_{EX}(F(G(x)),F(G(y))$ is the constant map).

\textbf{(iv)} Let us denote the functor sending $X$ to its associated discrete category by $\disc$. We write $C\colon\C\to\set$ for the constant functor sending each $X\mapsto *$. We will show that the functor $\hom_\cat(\C,\disc(-))$ is represented by $\pi_0(\C)$ for all $\C\in\ob(\cat)$. First of all, we define a map
\[
\Phi\colon\hom_\set(\pi_0(\C),S)\to\hom_\cat(\C,\disc(S))
\]
by letting for every $F\colon\pi_0(\C)\to S$ 
\begin{itemize}
\item $\ob(\Phi(F))\colon\ob(\C)\to S$, $X\mapsto F\circ\iota_X(*)$, and

\item $\hom_\C(X,Y)\to\hom_{\disc(S)}(\Phi X,\Phi Y)$ be $\left\{\begin{array}{ll}
\varnothing,&\text{ if }\Phi X\neq\Phi Y\\
\{\id\},&\text{ if }\Phi X=\Phi Y,
\end{array}\right.$
\end{itemize}
where $\iota\colon C\to \pi_0(\C)_\C$ is the coprojection.
\[
\begin{tikzcd}
C(X)=*\arrow[dr,"\iota_X"]\arrow[drrr,bend left=15,"*\mapsto G(X)"]\arrow[dd,"\id_*"']&&&\\
&\ilim_\C C\arrow[rr,dotted,"\Psi(G)"]&& S\\
C(Y)=*\arrow[ur,"\iota_Y"']\arrow[urrr,bend right=15,"*\mapsto G(Y)"']&&&
\end{tikzcd}
\]
Next we intend to define an inverse $\Psi$ to $\Phi$. For any functor $G\colon\C\to\disc(S)$, note that $G(X)=G(Y)$ if there is a morphism $X\to Y$ in $\C$. From this we get a cocone $C\to S_\C$ with $C(X)\to S$ sending $*\mapsto G(X)$, which defines a unique map $\ilim_\C C\to S$ via the universal property of colimits and we denote it by $\Psi(G)$. 

To see that $\Psi$ and $\Phi$ are mutually inverse, we have
\[
\Phi\circ\Psi(G)(X)=\Psi(G)\circ\iota_X(*)=G(X)
\]
for all $X\in\ob(\C)$ and $G\colon\C\to\disc (S)$, and
\[
(\Psi\circ\Phi(F))\circ\iota_X(*)=(\Psi(X\mapsto F\circ\iota_X(*)))\circ\iota_X(*)=F\circ\iota_X(*)
\]
for all $X\in\ob(\C)$ and $F\colon\pi_0(\C)\to S$. Therefore $\Psi\circ\Phi=\id$. Also, since the target of $\Phi\circ\Psi(G)$ is $\disc(S)$, in which the hom sets are either $\varnothing$ or $\id$, we have $\Phi\circ\Psi=\id$. 

As for the functoriality, one has the following commutative diagram
\[
\begin{tikzcd}
\hom_\set(\pi_0(\C),S)\arrow[r,equal,"\sim"]\arrow[d]& \hom_\cat(\C,\disc(S))\arrow[d]\\
\hom_\set(\pi_0(\C),S')\arrow[r,equal,"\sim"]& \hom_\cat(\C,\disc(S'))
\end{tikzcd}
\ 
\begin{tikzcd}[column sep=small]
F\arrow[rr,mapsto]\arrow[d,mapsto]&& \Phi(F)\arrow[d,mapsto]\\
s\circ F\arrow[r,mapsto]& \Phi(s\circ F)\arrow[r,equal]& \disc(s)\circ\Phi(F)
\end{tikzcd}
\]
for any map $s\colon S\to S'$ of sets. Here the equality is because
\[
\Phi(s\circ F)(X)=(s\circ F)\circ \iota_X(*)
\]
and
\[
\disc(s)\circ\Phi(F)(X)=s\circ\ob(\Phi(F))(X)=s\circ(F\circ\iota_X(*))
\]
for all $X\in\ob(\C)$.

\textbf{(v)} For a groupoid $\G$, $\pi_0(\G)$ is the set of isomorphism classes of $\G$. This can be seen by verifying the universal property of colimits. For the moment we denote by $\pi'_0(\G)$ the set of isomorphism classes. Define the coprojections $\iota_X\colon C(X)\to\pi_0'(\G)$ by sending $*\mapsto[X]$ (the isomorphism class of $X\in\ob(\G)$). Suppose that we have a cocone $F\colon C\to S_\G$ for some small set $S$. Then we can define a map
\[
f\colon\pi'_0(\G)\to S
\]
by $[X]\mapsto F_X(*)$. This is well-defined, since $F_X=F_Y\circ\id_*$ whenever $X\cong Y$. Such $f$ is unique, since if there is another $f'\colon\pi_0'(\G)\to S$, then 
\[
f'([X])=f'\circ \iota_X(*)=F_X(*)=f\circ\iota_X(*)=f([X])
\]
for all $X\in\ob(\G)$. This shows $\pi_0'(\G)\cong\pi_0(\G)$.

\end{document}

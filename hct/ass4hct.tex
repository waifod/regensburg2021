\documentclass[a4paper,11pt,openany]{scrartcl}
\usepackage{../reg2021}
\usepackage{../quiver}

\begin{document}
\noindent\textbf{Authors}\hfill\textbf{Semester} \linebreak
\vspace*{-.1cm} Matteo Durante, Yuhao Zhang\hfill WS 2020/21 \\

\noindent
\rule{\linewidth}{1pt}
\begin{center}
\Large
\textbf{Higher Category Theory} \\
Assignment 4
\end{center}
\rule{\linewidth}{1pt}
\\

%%%%%%%%%%%%%%%%%%%%%%%%%%%%%%%%%%%%%%%%%%%%%%%%%%%%%%%%%%%%%%%%%%%%%%%%%%%%%%%

\newcommand{\La}{\Lambda}
\newcommand{\pa}{\partial}
\newcommand{\ob}{\operatorname{Ob}}
\newcommand{\mor}{\operatorname{Mor}}
\newcommand{\sto}{\twoheadrightarrow}

\exercise{1}
\begin{proof}
\textbf{(1)} The inclusion maps induce functors 
\[
\tau(\La^3_k)\to\tau(\pa\Delta^3)\to\tau(\Delta^3)=\tau N([3])\cong[3].
\]
Recall that 
\[
\pa\Delta^3([i])=\{f\colon[i]\to[3]\mid\text{$f$ is not surjective}\}
\]
and it follows that $\Delta^3([i])=\pa\Delta^3([i])$ for $i<3$. Hence $Sk_2(\pa\Delta^3)=Sk_2(\Delta^3)$. Since $\tau(X)\cong\tau(Sk_2(X))$ for any simplicial set $X$, we get $\tau(\pa\Delta^3)\cong\tau(\Delta^3)$ by the functoriality of $Sk_2$. It remains to check that $\tau(\La^3_k)\to\tau(\Delta^3)$ (via composition) is an isomorphism for $k=1,2$. To this end, by the construction of $\tau$, we can depict $\tau(\La^3_1)$ as
\[
\begin{tikzcd}[sep=small]
& 1\arrow[dr]&\\
0\arrow[ur]\arrow[dr]\arrow[rr]&& 3\\
& 2\arrow[ur]\arrow[from=uu,crossing over]&
\end{tikzcd}
\]
where each face is commutative except possibly the face opposite to $1$, and the functor $\tau(\La^3_1)\to\tau(\Delta^3)$ sends $0\to 3$ and the composition $0\to 2\to 3$ to the same morphism $0\to 3$ in $\tau(\Delta^3)$. However, we claim that the face $0-2-3$ in $\tau(\La^3_1)$ is actually commutative. Indeed, the composition $0\to 2\to 3$ equals to the composition $0\to 1\to 2\to 3$, which in turns equals to $(0\to 1\to 3)=(0\to 3)$. The proof for $k=2$ is similar, only to focus on the face opposite to $2$. Combining the obtained isomorphisms yields the desired ones $\tau(\La^n_k)\cong\tau(\pa\Delta^3)\cong\tau(\Delta^3)$.\\
\\
\textbf{(2)} Recall that for any simplicial set $X$, $\tau(X)$ is defined by 
\begin{itemize}
\item $\ob(\tau(X))=X_0$, and

\item $\mor(\tau(X))=\{\Delta^1\to X\}/\text{commutative triangles}$.
\end{itemize}   
Also, note that
\[
\La^n_k([1])=\big\{f\colon[1]\to[n]\mid \im(f)\not\supseteq\{0,\cdots,k-1,k+1,\cdots,n\}\big\}
\]
and thus we have $\mor(\tau(\La^n_k))=\mor(\tau(\Delta^n))$ for $n\geqslant 3$; $(0\mapsto 1,1\mapsto2)\not\in\mor(\tau(\La^2_0))$, $(0\mapsto0,1\mapsto1)\not\in\mor(\tau(\La^2_2))$ for $n=2$; $(0\mapsto0,1\mapsto1)\not\in\mor(\tau(\La^1_0))\text{ and }\mor(\tau(\La^1_1))$. This shows that
\[
\tau(\La^n_0)=\left\{\begin{array}{ll}
\tau(\Delta^n)=[n]& n\geqslant3\\
\begin{tikzcd}[sep=tiny]
& 1&[-5pt]\\
0\arrow[ur]\arrow[rr]&& 2
\end{tikzcd}& n=2\\
0\quad1& n=1\\
0& n=0
\end{array}\right.\text{ and }
\tau(\La^n_n)=\left\{\begin{array}{ll}
\tau(\Delta^n)=[n]& n\geqslant3\\
\begin{tikzcd}[sep=tiny]
&[-5pt] 1\arrow[dr]&\\
0\arrow[rr]&& 2
\end{tikzcd}& n=2\\
0\quad1& n=1\\
0& n=0
\end{array}\right.
\]
Here all identity morphisms are omitted.\\
\\
\textbf{(3)} By definition, we must show that $C$ is a groupoid if and only if 
\[
\Hom_{\sSet}(\La^n_k,N(C))\sto\Hom_{\sSet}(\Delta^n,N(C))
\]
for all $n\geqslant1$ and all $0\leqslant k\leqslant n$. Since $C$ is a category, the epimorphism holds for $n\geqslant2$ and $0<k<n$, and hence one only needs to check the case $k=0,n$. Consider the adjunction
\[
\begin{tikzcd}
\Hom_{\sSet}(\La^n_k,N(C))\arrow[r,equal,"\sim"]& \Hom_{\Cat}(\tau\La^n_k,C)\\
\Hom_{\sSet}(\Delta^n,N(C))\arrow[u]\arrow[r,equal,"\sim"]& \Hom_{\Cat}(\tau\Delta^n,C)\arrow[u]
\end{tikzcd}
\]
and the left vertical arrow is epic if and only if the right vertical arrow is so. Thus far we reduced the case to proving that $C$ is a groupoid if and only if
\[
\Hom_{\Cat}(\tau\Delta^n,C)\sto\Hom_{\Cat}(\tau\La^n_k,C)\tag{$*$}
\]
for all $n\geqslant1$ and $k=0,n$. Moreover, by (2), it suffices to check the case of $n=2$.

To this end, if $(*)$ is epic for $k=0$ and $n$, then the diagrams
\[
\begin{tikzcd}
\tau\La^2_0\arrow[r]\arrow[d]& C\\
\text{[2]}\arrow[ur,dotted]
\end{tikzcd}
\text{ and }
\begin{tikzcd}
\tau\La^2_2\arrow[r]\arrow[d]& C\\
\text{[2]}\arrow[ur,dotted]
\end{tikzcd}
\]
admit extensions. In particular, we take $\tau\La^2_0\to C$ sending $0\to 2$ to some $\id_x$ and $0\to 1$ to any morphism $f\colon x\to y$. Then the extension provides a left inverse 
\[
\begin{tikzcd}[sep=small]
& y\arrow[dr,dotted,"g"]&\\
x\arrow[rr,"\id_x"']\arrow[ur,"f"]&& x
\end{tikzcd}
\]
On the other hand, we take $\tau\La^2_2\to C$ sending $0\to 2$ to $\id_x$ and $1\to 2$ to $f$, then the extension gives a right inverse $g'$ to $f$, which means that $f$ is an isomorphism. So $C$ is a groupoid.

Conversely, if $C$ is a groupoid, for any $\tau\La^2_0\to C$, which is given by some
\[
\begin{tikzcd}[sep=small]
& y&\\
x\arrow[rr,"g"']\arrow[ur,"f"]&& z
\end{tikzcd}
\]
we can invert $f$ to define $y\to z$, and this in turn gives rise to a functor $[2]\to C$ extending $\tau\La^2_0\to C$. This shows that $(*)$ is surjective for $k=0$. Analogously we can perform the same argument for $k=n$.
\end{proof}

~\\
\exercise{2}
\begin{proof}
    \textbf{(1)} Since $\Set$ is locally small, we may check that $(\cA_1,\cB_1)$ is a weak
    factorization system and $\cA_1$ is the smallest saturated class containing
    $I=\{0\rightarrow 1,2\rightarrow 1\}$ by applying the small object argument
    to $I$ itself and showing that $\cA_1=l(r(I))$. Indeed, $\Set(0,-)$ is the
    constant diagram at $1$ by initiality of
    $0$ and therefore,
    for any filtered diagram $D\colon\cI\rightarrow\Set$ (i.e.\ a functor whose
    indexing category is small and filtered), we get
    $\Set(0,\colim_{\cI}Di)=1=\colim_{\cI}1=\colim_\cI\Set(0,Di)$. Also,
    since $\Set(1,-)\cong\Id_{\Set}$, $D\cong\Set(1,D-)$ and therefore the colimit
    is trivially preserved. It follows that the small object argument applies
    and $l(r(I))$ is the smallest saturated class containing $I$.

    Let's fix a function $f\colon X\rightarrow Y$. For any element $y\in Y$, we
    may construct the following commutative diagram.
    \[
        \begin{tikzcd}
            0\ar[r]\ar[d]
            & X\ar[d, "f"] \\
            1\ar[r, "y"]
            & Y
        \end{tikzcd}
    \]
    We see that there exists an element $x\in X$ such that $f(x)=y$ if and only
    if there exists a function $x\colon 1\rightarrow X$ filling the diagram.
    Since every commutative square with these vertical arrows has this form, we
    have that $f$ is surjective if and only if it has the right lifting property
    with respect to $0\rightarrow 1$.

    Consider now $y\in Y$ and a commutative square
    \[
        \begin{tikzcd}
            2\ar[r, "g"]\ar[d]
            & X\ar[d, "f"] \\
            1\ar[r, "y"]
            & Y
        \end{tikzcd}
    \]
    For such a square to exists we need a $g$ with $f(g(*_0))=f(g(*_1))=y$, that
    is its image must be mapped to $y$ under $f$ and therefore $y\in\im(f)$.
    A filling is a choice of an element $x\in X$ such that $f(x)=y$ and
    $x=g(*_0)=g(*_1)$.

    \noindent If $f$ is injective then we have a unique $x\in X$ mapped
    to $y$, thus there is a unique $g$ making the diagram commute and a filling
    $x\colon 1\rightarrow X$. On the other hand, if it is not injective we can
    choose a $y\in Y$ such that $y=f(x_0)=f(x_1)$, $x_0\neq x_1$, and define $g$
    as $g(*_i)=x_i$, which with $y\colon 1\rightarrow Y$ will create a
    commutative diagram not admitting a filler. It follows that $f$ has the
    right lifting property with respect to $2\rightarrow 1$ if and only if it is
    injective.

    By what we have shown, $f\in r(I)$ if and only if it is bijective
    and therefore an isomorphism, thus $r(I)=\cB_1$.

    Consider now a function $g\colon X\rightarrow Y$, $f\in r(I)$ and a
    commutative diagram
    \[\begin{tikzcd}
	{X} & {S} \\
	{Y} & {T}
	\arrow["{g}"', from=1-1, to=2-1]
	\arrow["{f}", from=1-2, to=2-2]
	\arrow["{q}"', from=2-1, to=2-2]
	\arrow["{p}", from=1-1, to=1-2]
    \end{tikzcd}\]
    We can construct a filler by setting $h:=f^{-1}\cdot q$ since $h\cdot
    g=f^{-1}\cdot q\cdot g=f^{-1}\cdot f\cdot p=p$ and $f\cdot h=f\cdot
    f^{-1}\cdot q=q$, hence $g\in l(r(I))$ and $\cA_1=l(r(I))$.

    This in particular shows that $\cA_1$ and $\cB_1$ are saturated classes. We
    want to prove that $(\cB_1,\cA_1)$ is a weak factorization system as well.

    Given a function $f\colon X\rightarrow Y$, we see that $f=f\cdot\id_X$,
    where $\id_X\in\cB_1$, $f\in\cA_1$, while looking at the previous
    commutative square and supposing that $g\in\cB_1$, $f\in\cA_1$, we get a
    filler by considering $h:=p\cdot g^{-1}$, thus $\cB_1\subset l(\cA_1)$ and
    we have the thesis.\\
    \\
     \textbf{(2)} As we have said earlier, $\Set$ is locally small and $\Set(0,-)$
    preserves filtered colimits, thus setting $I=\{0\rightarrow 1\}$ and
    applying the small object argument we get that $(l(r(I)),r(I))$ is a weak
    factorization system and $l(r(I))$ is the smallest saturated class in $\Set$
    containing $I$.

    By what we have shown in (1), $r(I)$ is the class of all surjective
    functions. Let's consider a commutative square
    \[
        \begin{tikzcd}
            X\ar[d, "f"]\ar[r, "p"]
            & S\ar[d, "g"] \\
            Y\ar[r, "q"]
            & T
        \end{tikzcd}
    \]
    where $g\in r(I)$. To construct a filler $h\colon Y\rightarrow S$ we need to
    pick for every $y\in Y$ an element $h(y)\in g^{-1}(q(y))$ in such a way that,
    whenever $y=f(x)$, we also have $h(y)=p(x)$.

    \noindent If $f$ is injective, then we
    set $h(f(x)):=p(x)$ for all $x\in X$, while for all $y\in Y\setminus \im(f)$
    we choose freely $h(y)$ from $g^{-1}(q(y))$ and this constitutes a filler.

    \noindent On the other hand, if it is not
    injective, then there are two distinct elements $x_0,x_1\in X$ such that
    $y=f(x_0)=f(x_1)$ and we may consider the surjection $2\rightarrow 1$ as
    $g$, the unique map $Y\rightarrow 1$ as $q$ and pick $p$ such that
    $p(x_i)=*_i$ and the square commutes. A filling $h\colon Y\rightarrow 2$
    would have to satisfy $h(f(x_0))=h(f(x_1))$ and $h(f(x_i))=p(x_i)=*_i$,
    which is absurd.

    \noindent It follows that $l(r(I))$ is the class of all injective functions.\\
    \\
     \textbf{(3)} Once again, the small object argument applies with $I=\{1\rightarrow
    2\}$. Consider a function $f\colon X\rightarrow Y$. If $X=0$, since there
    are no functions $1\rightarrow 0$, there are no commutative squares with $f$
    on the right and $1\rightarrow 2$ on the left, hence $f\in r(I)$ trivially.
    Suppose now $X\neq 0$. A commutative square
    \[\begin{tikzcd}
	{1} & {X} \\
	{2} & {Y}
	\arrow["{x}", from=1-1, to=1-2]
	\arrow["{q}", from=2-1, to=2-2]
	\arrow[from=1-1, to=2-1]
	\arrow["{f}", from=1-2, to=2-2]
    \end{tikzcd}\]
    is given by a choice of a pair $(x,y)\in X\times Y$, where $x$ defines the
    upper map and $q(*_0):=f(x)$, $q(*_1):=y$. A filling $h\colon 2\rightarrow
    X$ then exists if and only if there exists $x'\in X$ such that
    $f(x')=q(*_1)$, in which case $h(*_0)=x$, $h(*_1)=x'$. Asking for all the
    fillings to exist is equivalent to saying that $f$ is surjective.

    It follows that $r(I)$ is the class of functions which are either surjective
    or have empty domain.

    We now have to compute $l(r(I))$. Let's consider a function $g\colon
    S\rightarrow T$. If $T=0$, then $g=\id_0$ and it has the left lifting
    property against any function thanks to the initiality of 0. If $S\neq 0$,
    then the only lifting problems we have to consider are the ones where the
    function $f$ on the right is surjective and has a non-empty domain. By an
    argument provided in (1) using $2\rightarrow 1$, we see that such a $g$ must
    be injective. Finally,
    if $S=0$, $T\neq 0$ we have for any pair of functions $f\colon
    X=0\rightarrow Y$, $q\colon T\rightarrow Y$ a commutative square
    \[\begin{tikzcd}
	{0} & {0} \\
	{T} & {Y}
	\arrow[from=1-1, to=1-2]
	\arrow["{q}", from=2-1, to=2-2]
    \arrow["{g}", from=1-1, to=2-1]
	\arrow["{f}", from=1-2, to=2-2]
    \end{tikzcd}\]
    which does not admit a filling, hence $g\not\in l(r(I))$.

    It follows that $l(r(I))$ is the class of functions which are
    injective and have either a non-empty domain or an empty codomain.\\
    \\
     \textbf{(4)} Consider a weak factorization system $(\cA,\cB)$ and remember that
    $\cA=l(\cB)$ implies that $\cA$ is saturated and with $\cB=r(\cA)$ means
    that is enough to determine one of them. We will show that it
    falls in one of the cases we have already studied.

    We begin by noticing that any bijection
    lies in $\cA\cap\cB$ since it has the right and left lifting property with
    respect to every map, as shown in (1).

    Since any function $f$ admits a factorization
    $p\cdot i$, where $i\in\cA$, $p\in\cB$, we have $\cA,\cB\neq\emptyset$. In
    particular, for the injection $0\rightarrow 1$ we have $i\colon 0\rightarrow
    X$.

    Let's focus on $i$ and suppose $X\neq 0$. Then we have a retraction
    \[\begin{tikzcd}
	{0} & {0} & {0} \\
	{1} & {X} & {1}
	\arrow[from=1-1, to=1-2]
	\arrow[from=1-2, to=1-3]
	\arrow[from=1-1, to=2-1]
	\arrow[from=1-2, to=2-2]
	\arrow[from=1-3, to=2-3]
	\arrow[from=2-2, to=2-3]
	\arrow[from=2-1, to=2-2]
    \end{tikzcd}\]
    which implies that $0\rightarrow 1$ lies in $\cA$ and therefore
    $\cA_2\subset\cA$, $\cB\subset\cB_2$.

    If $\cA$ does not contain any non-injective function, then $\cA=\cA_2$ and
    $\cB=\cB_2$, that is we are in case (2). On the other hand, if it does
    contain a non-injective function
    $g\colon S\rightarrow T$, consider $s_0,s_1\in S$ such that
    $t=g(s_0)=g(s_1)$. Constructing $f\colon 2\rightarrow S$ with $f(*_i)=s_i$
    and taking a retraction $r\colon S\rightarrow 2$, we get the commutative
    diagram
    \[\begin{tikzcd}
	{2} & {S} & {2} \\
	{1} & {T} & {1}
	\arrow["f", from=1-1, to=1-2]
	\arrow["r", from=1-2, to=1-3]
	\arrow[from=1-1, to=2-1]
	\arrow["g", from=1-2, to=2-2]
	\arrow[from=1-3, to=2-3]
	\arrow[from=2-2, to=2-3]
	\arrow["t", from=2-1, to=2-2]
    \end{tikzcd}\]
    exhibiting $2\rightarrow 1$ as a retract of $g$, in which case $\cA_1=\cA$
    and therefore $\cB=\cB_1$, hence we are in case (1.a).

    Suppose instead that a factorization of $0\rightarrow 1$ where $X\neq 0$
    does not exist. Then, $0\rightarrow 1$ lies in $\cB$. We want to show that
    we are in case (1.b) or (3).

    By the argument provided in (3), given a function $g$, having the left
    lifting property with respect to $0\rightarrow 1$ implies that either the
    codomain is empty (i.e.\ $g=\id_0$) or the domain is non-empty.

    Suppose $\cB_1\subsetneq\cA$. If all of the maps in $\cA$ are injective,
    then $\cA\subset\cA_3$ and there exists a function $g\colon
    S\rightarrow T$ such that $S\neq 0$ and $g$ is injective but not surjective,
    hence we may take $t_0,t_1\in T$ such that $g(s)=t_0$ for some $s\in S$ and
    $t_1\not\in\im(g)$. We may now define $q\colon 2\rightarrow T$ as
    $q(*_i):=t_i$ and take the retraction $r$ given by $r(g(s))=*_0$, $r(t)=*_1$
    for $t\in T\setminus\im(g)$, which allows us to construct the
    commutative diagram
    \[\begin{tikzcd}
	{1} & {S} & {1} \\
	{2} & {T} & {2}
	\arrow["{s}", from=1-1, to=1-2]
	\arrow[from=1-2, to=1-3]
	\arrow["{q}", from=2-1, to=2-2]
	\arrow["{r}", from=2-2, to=2-3]
	\arrow[from=1-1, to=2-1]
	\arrow["{g}", from=1-2, to=2-2]
	\arrow[from=1-3, to=2-3]
    \end{tikzcd}\]
    exhibiting $1\rightarrow 2$ as a retract of $g$, proving that it lies in
    $\cA$. This gives us $\cA_3\subset\cA$, thus $\cA=\cA_3$.

    If $\cA$ contains a non-injective map $g\colon S\rightarrow T$, then then we
    may proceed as we have already done and get that $2\rightarrow 1$
    itself lies in $\cA$ as its
    retraction, which implies that functions in $\cB$ are injective. If there is
    one $f\colon X\rightarrow Y$ in $\cB$ such that $X\neq 0$ and it is not a
    bijection, then by a previous construction we get $1\rightarrow 2$ as a
    retraction.

    Consider a non-surjective map $g\colon S\rightarrow T$. We want to show that
    it can't lie in $\cA$. Clearly $T\neq\emptyset$, hence if it belonged to
    this class it would have $S\neq\emptyset$ and we can restrict ourselves to
    this case. We can
    construct a commutative square
    \[\begin{tikzcd}
	{S} & {1} \\
	{T} & {2}
	\arrow[from=1-1, to=1-2]
	\arrow["{q}", from=2-1, to=2-2]
	\arrow[from=1-2, to=2-2]
	\arrow["{g}"', from=1-1, to=2-1]
    \end{tikzcd}\]
    where $q(g(s))=*_1$ and $q(t)=*_2$ for all other $t\in T$. By construction,
    this square does not admit a filling $h$ since $q$ is not constant. It
    follows that all maps in $\cA$ are surjective.
\end{proof}

\end{document}

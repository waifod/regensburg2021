\documentclass[a4paper,11pt,openany]{scrartcl}
\usepackage{../reg2021}
\usepackage{../quiver}

\begin{document}
\noindent\textbf{Author}\hfill\textbf{Semester} \linebreak
\vspace*{-.1cm} Matteo Durante, Yuhao Zhang\hfill WS 2020/21 \\

\noindent
\rule{\linewidth}{1pt}
\begin{center}
\Large
\textbf{Higher Category Theory} \\
Assignment 12
\end{center}
\rule{\linewidth}{1pt}
\\

%%%%%%%%%%%%%%%%%%%%%%%%%%%%%%%%%%%%%%%%%%%%%%%%%%%%%%%%%%%%%%%%%%%%%%%%%%%%%%%

%%%sheet4
\newcommand{\La}{\Lambda}
\newcommand{\pa}{\partial}
\newcommand{\ob}{\operatorname{Ob}}
\newcommand{\mor}{\operatorname{Mor}}
\newcommand{\sto}{\twoheadrightarrow}

%%%sheet5
\newcommand{\plim}{\varprojlim}
\newcommand{\sst}{\subseteq}
\newcommand{\eq}{\operatorname{eq}}

%%%sheet6
\newcommand{\f}{\varphi}

%%%sheet8
\newcommand{\sing}{\operatorname{Sing}}

%%%sheet9
\newcommand{\ihom}{\underline{\Hom}}

%%%sheet11
\newcommand{\N}{\mathbb{N}}


~\\
\exercise{1}
\begin{proof}
    (1) Objects in $A/F$ are natural transformations $t\colon\yo_a\rightarrow
    F$, that is elements $t\in Fa$ for some $a\in A$, while morphisms $f\colon
    t\rightarrow t'$ are natural transformations $f^*\colon
    \yo_a\rightarrow\yo_{a'}$ such that the triangle
    \[
        \begin{tikzcd}
            h_a\ar["f^*", rr]\ar["t"', dr]
            && h_{a'}\ar["t'", dl] \\
            & F
        \end{tikzcd}
    \]
    commutes, or equivalently morphisms $f\colon a\rightarrow a'$ in $A$ with
    $F(f)(t')=t$ by Yoneda (we will be abusing the notation by referring to the
    induced morphisms in the slice category by the same names as the original
    ones).

    After fixing an object $t$ in $A/F$, we get the functor $\Pi_F\colon
    (A/F)/t\rightarrow A/a$, $(f\colon u\rightarrow t)\mapsto (f\colon
    b\rightarrow a)$.

    \noindent We start by proving that it is a bijection on objects,
    for which we fix a $f\colon b\rightarrow a$ in $A/a$ and try to construct
    an object $g\colon u\rightarrow t$ in $(A/F)/t$ mapped to it while proving
    its uniqueness. Remember that this object is induced by a natural
    transformation between representable presheaves $\yo_b\rightarrow\yo_a$,
    for which we simply take $f_*\colon\yo_b\rightarrow\yo_a$, coming from
    $f\colon b\rightarrow a$ in $A$.
    Our previous description tells us that this is indeed the desired map. For
    uniqueness, remember that the Yoneda embedding is fully faithful and
    therefore there is a bijection between natural trasformation amongst
    representable presheaves and morphisms in $A$.
    \[
        \begin{tikzcd}
            \yo_b\ar["f_*", rr]\ar["u"', dr]
            && \yo_a\ar["t", dl] \\
            & F
        \end{tikzcd}
    \]

    Observe that $\Pi_F$ is naturally faithful since distinct parallel
    morphisms $f,f'$ are given by definition by distinct morphisms $f,f'$ in
    $A/F$, which are themselves induced by distinct natural transformations
    between representable presheaves coming from distinct morphisms in $A$. The
    images of $f,f'$ under $\Pi_F$ are precisely the morphisms in $A/a$ induced
    by these morphisms in $A$ and are therefore distinct by construction.

    For fullness, consider two objects $g\colon u\rightarrow t,h\colon
    v\rightarrow t$ in $(A/F)/t$ and a morphism $f\colon
    g\rightarrow h$ in $A/a$ induced by $f\colon c=\dom h\rightarrow b=\dom g$.
    We simply have to prove that $f$ induces a morphism $g\rightarrow h$ in
    $(A/F)/t$. To do this, consider the diagram
    \[\begin{tikzcd}
        {\yo_c} && {\yo_a} \\
        {\yo_b} && F
        \arrow["{h_*}", from=1-1, to=1-3]
        \arrow["t", from=1-3, to=2-3]
        \arrow["v"', from=2-1, to=2-3]
        \arrow["{f_*}"', from=1-1, to=2-1]
        \arrow["{g_*}"{description, pos=0.2}, from=2-1, to=1-3]
        \arrow["u"{description, pos=0.2}, crossing over, from=1-1, to=2-3]
    \end{tikzcd}\]
    and observe that
    \begin{align*}
        v\cdot f_* &=t\cdot g_*\cdot f_* \\
        &=t\cdot h_* \\
        &=u
    \end{align*}
    proving that $f$ does define a morphism $u\rightarrow v$ in $A/F$. Since
    $h=g\cdot f$, we can conclude that $f$ does define the desired morphism
    $h\rightarrow g$ and, under $\pi_F$, it is mapped to $f$ itself, proving
    fullness.

    (2) Consider a natural transformation $\alpha\colon F\Rightarrow G$. We can
    define a functor $\psi(\alpha)\colon A/F\rightarrow A/G$ (here called $\phi$
    for brevity) as
    $(t\colon\yo_a\Rightarrow F)\mapsto (\alpha_a(t)=\alpha\cdot
    t\colon\yo_a\Rightarrow G)$ on objects, $f\mapsto f$ on morphisms. It
    truly is a functor since it is well defined and identities and compositions
    are trivially preserved. Also, we see that $\dom t=\dom\alpha_a(t)$, which
    since $\phi$ is an identity on morphisms implies that
    $\pi_G\cdot\phi=\pi_F$. We only have to prove that this is a bijection.

    We will do this by constructing a natural transformation $\beta(\phi)\colon
    F\Rightarrow G$ (here called $\alpha$ for brevity) from a functor
    $\phi\colon A/F\rightarrow A/G$ such that
    $\pi_G\cdot\phi=\pi_F$. To do this, consider an object $a$ in $A$ and an
    element $t\in Fa$. This corresponds to a natural transformation
    $t\colon\yo_a\Rightarrow F$ which under $phi$ is sent to another natural
    transformation $\phi(t)\colon\yo_b\Rightarrow A/G$. Observe that
    \begin{align*}
        b &=\pi_G(\phi(t)) \\
        &=\pi_F(t) \\
        &=a, \\
        \pi_G(\phi(f)) &=\pi_F(f) \\
        &=f,
    \end{align*}
    where $f$ denotes a morphism in $A$ and the corresponding ones in $A/F$ and
    $A/G$. This means that the domains of the objects in the slices are
    preserved by $\phi$ and so are the morphisms, allowing us to set
    $\alpha_a(t)=\phi(t)$.

    We only still have to check for naturality and for this first we take a
    morphism $f\colon a\rightarrow b$ in $A$ and observe that $f\colon
    t\rightarrow u$ in $A/F$ is sent to $f\colon\phi(t)\rightarrow\phi(u)$,
    giving us the diagrams
    \[\begin{tikzcd}
        {\yo_a} && {\yo_b} \\
        & F
        \arrow["t"', from=1-1, to=2-2]
        \arrow["u", from=1-3, to=2-2]
        \arrow["{f_*}", from=1-1, to=1-3]
    \end{tikzcd}
    \quad
    \begin{tikzcd}
        {\yo_a} && {\yo_b} \\
        & G
        \arrow["\phi(t)"', from=1-1, to=2-2]
        \arrow["\phi(u)", from=1-3, to=2-2]
        \arrow["{f_*}", from=1-1, to=1-3]
    \end{tikzcd}\]
    from which we derive
    \begin{align*}
        (\alpha_b\cdot Ff)(t) &=\alpha_b(Ff(t)) \\
        &=\phi(Ff(t)) \\
        &=\phi(t\cdot f_*) \\
        &=\phi(t)\cdot f_* \\
        &=Gf(\phi(t)) \\
        &=Gf(\alpha_a(t)) \\
        &=(Gf\cdot\alpha_a)(t)
    \end{align*}
    We are left with checking that these associations are inverse to one
    another.

    Fix then $\alpha\colon F\Rightarrow G$ and pick $t\in Fa$. We have
    \begin{align*}
        (\beta_a(\psi(\alpha)))(t) &=(\psi(\alpha))(t) \\
        &=\alpha\cdot t \\
        &=\alpha_a(t)
    \end{align*}
    which gives us $\beta\cdot\psi=\id$. Also, fixing a functor $\phi\colon
    A/F\Rrightarrow A/G$ and picking an object $t\in Fa$ and a morphism $f\colon
    t\rightarrow u$, we see that
    \begin{align*}
        (\psi(\beta(\phi)))(t) &=\beta(\phi)_a\cdot t \\
        &=(\beta(\phi)_a)(t) \\
        &=\phi(t),
        (\psi(\beta(\phi)))(f) &=f \\
        &=\phi(f),
    \end{align*}
    proving that $\psi\cdot\beta=\id$ and therefore the thesis.
\end{proof}

~\\
\exercise{2}
\begin{proof}
Suppose we have had $ho(A)^{\op}\to\Set$, $a\mapsto X_a$.

Next we construct a functor $ho(A)^{\op}\to\Set$, $a\mapsto\pi_0(\Fun(W,X_a))$ for each simplicial set $W$. To this end, note that $\Fun(W,X)\to\Fun(W,A)$ is a right fibration since $p\colon X\to A$ is so, and $\Fun(W,A)$ is an $\infty$-category since $A$ is so. Hence we get a functor 
\[
ho(\Fun(W,A))^{\op}\to\Set
\]
sending $f\mapsto\pi_0(\Fun(W,X)_f)$. On the other hand, let us apply the homotopy category functor to $A=\Fun(\Delta^0,A)\to\Fun(W,A)$ and get 
\[
ho(A)\to ho(\Fun(W,A)).
\]
So we obtain a functor $ho(A)^{\op}\to\Set$ sending $a\mapsto\pi_0(\Fun(W,X)_{\Fun(W,a)})$. Recall that $\Fun(W,-)$ is a right adjoint, applying which to the pullback diagram left-hand side below yields the pullback diagram on the right:
\[
\begin{tikzcd}
X_a\arrow[r]\arrow[d]\arrow[dr,phantom,"\lrcorner",very near start]& X\arrow[d]\\
\Delta^0\arrow[r]& A
\end{tikzcd}
\qquad\qquad
\begin{tikzcd}
\Fun(W,X_a)\arrow[r]\arrow[d]\arrow[dr,phantom,"\lrcorner",very near start]& \Fun(W,X)\arrow[d]\\
\Delta^0=\Fun(W,\Delta^0)\arrow[r,"\Fun(W\text{,}a)"]& \Fun(W,A)
\end{tikzcd}
\]
Thence $\Fun(W,X)_{\Fun(W,a)}=\Fun(W,X_a)$ and this leads to our prescribed functor $ho(A)^{\op}$ $\to\Set$.

Finally, we come back to show that $ho(A)^{\op}\to ho(\sSet)$, $a\mapsto X_a$ is well-defined. In fact, suppose that $a,b\in A_0$ and two $1$-simplex with source $a$ and target $b$ are homotopic. Then for every simplicial set $W$, their induced map $\pi_0(\Fun(W,X_b))\to\pi_0(\Fun(W,X_a))$ are the same by the previous paragraph. Therefore if $a$ and $b$ are homotopy equivalent, then we have a bijection (and by Exercise 1(3) of Sheet 9)
\[
[W,X_b]\cong\pi_0(\Fun(W,X_b))\cong\pi_0(\Fun(W,X_a))\cong[W,X_a].
\]
Hence $X_a$ and $X_b$ are (weak) homotopy equivalent, which shows the well-definedness on objects. For the well-definedness on morphisms, given two homotopic $1$-simplex from $a$ to $b$, it suffices to take $W=X_a$ in the induced maps $[W,X_b]\to[W,X_a]$ and so their induced $X_b\to X_a$ lie in the same homotopy class.
\end{proof}

~\\
\exercise{3}
\begin{proof}
By a lemma in Lecture 12 we have the following correspondence of lifting problems
\[\tag{$*$}
\begin{tikzcd}
\pa\Delta^n\arrow[r]\arrow[d]& X_{/x}\arrow[d]\\
\Delta^n\arrow[r]\arrow[ur,dotted]& Y_{/px}
\end{tikzcd}
\rightsquigarrow
\begin{tikzcd}
\pa\Delta^n*\Delta^0\arrow[r]\arrow[d]& X\arrow[d,"p"]\\
\Delta^n*\Delta^0\arrow[r]\arrow[ur,dotted]& Y
\end{tikzcd}
\]
We claim that $\pa\Delta^n*\Delta^0\cong\La^{n+1}_{n+1}$. In fact\footnote{\ Or we can just cite a proposition in Lecture 11, that $\Delta^n*\La^m_l\cup\pa\Delta^n*\Delta^m=\La^{n+1+m}_{n+1+l}$ as a simplicial subset of $\Delta^n*\Delta^m$, and take $m=l=0$.}, for every $[m]\in\Delta$ we have by definition
\[
(\pa\Delta^n*\Delta^0)_m=\coprod_{\substack{i+1+j=m\\ -1\leqslant i,j\leqslant m}}\pa\Delta^n_i\times\Delta^0_j=\pa\Delta^n_m\amalg(\pa\Delta^n_{m-1}\times\Delta^0_0)\cong\La^{n+1}_{n+1}([m]).
\]
where the last bijection is given by $\pa\Delta^n_m\to\La^{n+1}_{n+1}([m])$, $([m]\to[n])\mapsto([m]\to[n]\hookrightarrow[n+1])$ and $\Delta^n_{m-1}\times\Delta^0_0\to\La^{n+1}_{n+1}([m])$, $(f\colon[m-1]\to[n],*)\mapsto(g\colon[m]\to[n+1])$ such that $g(m)=n+1$, $g|_{[m-1]}=f$.

Therefore, if $p$ is a right fibration, then it admits a lift against $\La^{n+1}_{n+1}\to\Delta^{n+1}$ for all $n\geqslant0$ and hence the left-hand side of $(*)$ has a filler $\Delta^n\to X_{/x}$, so that $X_{/x}\to Y_{/px}$ is a trivial fibration. Conversely, if $X_{/x}\to Y_{/px}$ is a trivial fibration, then the right-hand side of $(*)$ admits a filler. With $p$ being also an inner fibration, we see that it has RLP against all $\La^n_k\to\Delta^n$ for $0<k\leqslant n$, and thus is a right fibration.
\end{proof}

\end{document}

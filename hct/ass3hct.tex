\documentclass[a4paper,11pt,openany]{scrartcl}
\usepackage{../reg2021}

\begin{document}

\title{Higher Category Theory\\ Assignment 3}

\author{Matteo Durante}

\maketitle

\exercise{1}
\begin{proof}
    (1) Let $g\colon b\rightarrow b'$ be such that $Gg$ is an isomorphism. Then
    there exists $f\colon Gb'\rightarrow Gb$ such that $Gg\cdot f=\id_{Gb'}$,
    $f\cdot Gg=\id_{Gb}$ and, since $G$ is full, we have
    $g'\in\cD(b',b)\cong\cC(Gb',Gb)$ such that $Gg'=f$. Having $G(g\cdot
    g')=Gg\cdot Gg'=Gg\cdot f=\id_{Gb'}$, $G(g'\cdot g)=Gg'\cdot Gg=f\cdot
    Gg=\id_{Gb}$, by faithfullness $g\cdot g'=\id_{b'}$, $g'\cdot g=\id_b$.

    (2) We will refer to the diagram mentioned as $D\colon\cI\rightarrow\cD$ in
    order to distinguish it from the functor $F$ defining the adjunction. Now,
    dualizing the proofs given in the solution of exercise 3 of the previous
    sheet, we see that the right adjoint $G$ is fully faithful if and only if
    the natural transformation $\epsilon\colon FG\Rightarrow\id_{\cD}$ induced
    by the adjunction is an isomorphism. Now, since left adjoints preserve
    colimits, taken the universal cocone $\lambda\colon GD\Rightarrow\colim_\cI
    GD$ we get another one $F\lambda\colon FGD\Rightarrow
    F\colim_{\cI}GD$. Composing with $\epsilon^{-1}D$, we get then a universal
    cocone $F\lambda\cdot\epsilon^{-1}D\colon D\Rightarrow
    F\colim_{\cI}GD$,
    which exhibits $F\colim_{\cI}GD\cong\colim_{\cI}FGD$ as the colimit of $D$.

    (3) We shall prove that $(F,G,\eta,\epsilon)$ defines a monadic adjunction,
    which will imply that $G$ creates limits. By (2), $\cD$ admits coequalizers
    of $G$-split pairs and the full faithfullness of $G$ implies reflexivity,
    which with $\cC$ having their colimits implies that $G$ preserves them.
    Also, by (1) we have conservativity, which allows us to apply Beck's theorem
    and conclude.

    We want to prove that fully faithful functors preserve limits (and dualizing
    colimits). We will start by proving that they reflect them.

    Consider $F\colon\cC\rightarrow\cD$ like that and a diagram
    $D\colon\cI\rightarrow\cC$ with a cone $\lambda\colon c\Rightarrow D$ such
    that $F\lambda\colon Fc\Rightarrow FD$ is universal. Then, given another
    cone $\alpha\colon c'\Rightarrow D$ we have that $F\alpha$ factors uniquely
    through $F\lambda$ as $g\colon Fc'\rightarrow Fc$. By full faithfullness, we
    have a unique $f\colon c'\rightarrow c$ such that $Ff=g$ and (again by full
    faithfullness) it factors $\alpha$ through $\lambda$. Also, any other
    factorization would be an arrow $c'\rightarrow c$ sent by $F$ to $g$, which
    implies that $f$ gives the only one.

    Consider now a diagram $D\colon\cI\rightarrow\cC$ admitting a universal cone
    $\lambda\colon c\Rightarrow D$ and such that $FD$ does too.
\end{proof}

~\\
\exercise{2}
\begin{proof}
    (1) Notice that such an endofunctor $\rho$ has to satisfy $\rho([n])=[n]$.
    Consider $\sigma^{n-1}_i$. We know that it is the left inverse of
    $\delta^n_i$ and $\delta^n_{i-1}$ (if $i>0$). From these considerations, we
    get that $\rho(\sigma^{n-1}_i)$ has to be the left inverse of
    $\rho(\delta^n_i)=\delta^n_{n-i}$ and
    $\rho(\delta^n_{i-1})=\delta^n_{n+1-i}$, which is enough to reconstruct it
    thanks to the injectivity of the right inverses and determine that it is
    precisely $\sigma^{n-1}_{n-i-1}$. This is enough to prove that, if such an
    endofunctor exists, then it is unique.

    One verifies that all of these associations preserve the desired relations
    and, since $\Delta$ is obtained by taking the free category generated by
    these arrows and then quotienting by the aforementioned equations, we get
    that $\rho$ does define an endofunctor $\Delta\rightarrow\Delta$, which one
    can verify to be an involution as it defines one on the morphisms
    generating the category. It follows that it also defines an involution
    $\rho^*\colon\sSet\rightarrow\sSet$. Also, notice that the functor $\rho$ is
    obtained simply by reversing the orderings of the elements of each $[n]$, so
    it acts on the simplices by ``inverting'' the faces.

    The isomorphism $\phi\colon N(\cC)^{\op}\rightarrow N(\cC^{\op})$ is given
    by sending $f\colon\Delta^1\rightarrow N(\cC)^{\op}$ to
    $\rho^*(f)^{\op}\colon\Delta^1\rightarrow N(\cC^{\op})$. Also, given a
    commutative triangle $(f,g,h)$ exhibited by a 2-simplex $t\rightarrow
    N(\cC)^{\op}$ in $N(\cC)^{\op}$, we see that applying
    $\rho^*$ turns it into another commutative triangle
    $(\rho^*(g),\rho^*(f),\rho^*(h))$ exhibited by $\rho^*(t)$. Looking at the
    description of $\rho^*$, we see that this actually corresponds to a
    commutative triangle in the category $\cC$ and it returns our starting
    triangle $(f,g,h)$ when we reapply $\rho^*$. But then, if
    $\rho^*(g)\cdot\rho^*(f)=\rho^*(h)$ in $\cC$, we get that
    $\rho^*(f)^{\op}\cdot\rho^*(g)^{\op}=\rho^*(h)^{\op}$ in $\cC^{\op}$.
    Similarly,
    $\rho^*(\id_x)^{\op}=\rho^*(s^0_0(x))^{\op}=s^0_0(\rho^*(x))^{\op}=\id_{\rho^*(x)}^{\op}=\id_x^{\op}$
    and therefore our natural transformation is well defined.
    We still have to check that it is an isomorphism. To do this we show that
    $N(\cC)^{\op}$ satisfies the Grothendieck-Segal condition and then we are
    done since the arrows are obtained by formally reversing the ones of
    $N(\cC)$, while our natural transformation is just reversing them twice and
    therefore it is essentially an identity on maps.
    \[
        \begin{tikzcd}
            \sSet(\Delta^n,N(\cC)^{\op})\ar[r]\ar[d, "\rho^*"]
            & \sSet(\Lambda_i^n,N(\cC)^{\op})\ar[d, "\rho^*"] \\
            \sSet(\Delta^n,N(\cC))\ar[r]
            & \sSet(\Lambda_{n-i}^n,N(\cC)) \\
        \end{tikzcd}
    \]
    The vertical arrows and the bottom one in this commutative diagram are
    isomorphisms for all $0<i<n$, hence the top one has to be an isomorphism
    too.

    (2) A similar proof applies to this case. Indeed, we may consider the
    commutative diagram
    \[
        \begin{tikzcd}
            \sSet(\Delta^n,X^{\op})\ar[r]\ar[d, "\rho^*"]
            & \sSet(\Lambda_i^n,X^{\op})\ar[d, "\rho^*"] \\
            \sSet(\Delta^n,X)\ar[r]
            & \sSet(\Lambda_{n-i}^n,X) \\
        \end{tikzcd},
    \]
    where the vertical arrows are isomorphisms and the bottom one is surjective
    for all $0<i<n$, which implies that the top one is surjective too.
\end{proof}

~\\
\exercise{3}
\begin{proof}
    (6) Notice that the diagram 
\end{proof}

\end{document}

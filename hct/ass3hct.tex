\documentclass[a4paper,11pt,openany]{scrartcl}
\usepackage{../reg2021}

\begin{document}

\title{Higher Category Theory\\ Assignment 3}

\author{Matteo Durante}

\maketitle

\exercise{1}
\begin{proof}
    (1) Let $g\colon b\rightarrow b'$ be such that $Gg$ is an isomorphism. Then
    there exists $f\colon Gb'\rightarrow Gb$ such that $Gg\cdot f=\id_{Gb'}$,
    $f\cdot Gg=\id_{Gb}$ and, since $G$ is full, we have
    $g'\in\cD(b',b)\cong\cC(Gb',Gb)$ such that $Gg'=f$. Having $G(g\cdot
    g')=Gg\cdot Gg'=Gg\cdot f=\id_{Gb'}$, $G(g'\cdot g)=Gg'\cdot Gg=f\cdot
    Gg=\id_{Gb}$, by faithfullness $g\cdot g'=\id_{b'}$, $g'\cdot g=\id_b$.

    (2) We will refer to the diagram mentioned as $D\colon\cI\rightarrow\cD$ in
    order to distinguish it from the functor $F$ defining the adjunction. Now,
    dualizing the proofs given in the solution of exercise 3 of the previous
    sheet, we see that the right adjoint $G$ is fully faithful if and only if
    the natural transformation $\epsilon\colon FG\Rightarrow\id_{\cD}$ induced
    by the adjunction is an isomorphism. Now, since left adjoints preserve
    colimits, taken the universal cocone $\lambda\colon GD\Rightarrow\colim_\cI
    GD$ we get another one $F\lambda\colon FGD\Rightarrow
    F\colim_{\cI}GD$. Composing with $\epsilon^{-1}D$, we get then a universal
    cocone $F\lambda\cdot\epsilon^{-1}D\colon D\Rightarrow
    F\colim_{\cI}GD$,
    which exhibits $F\colim_{\cI}GD\cong\colim_{\cI}FGD$ as the colimit of $D$.

    (3) We may prove that $(F,G,\eta,\epsilon)$ defines a monadic adjunction,
    which will imply that $G$ creates limits. By (2), $\cD$ admits coequalizers
    of $G$-split pairs and one has to prove that $G$ preserves them.
    Also, by (1) we have conservativity, which allows us to apply Beck's theorem
    and conclude.
\end{proof}

~\\
\exercise{2}
\begin{proof}
    (1) Notice that such an endofunctor $\rho$ has to satisfy $\rho([n])=[n]$.
    Consider $\sigma^{n-1}_i$. We know that it is the left inverse of
    $\delta^n_i$ and $\delta^n_{i-1}$ (if $i>0$). From these considerations, we
    get that $\rho(\sigma^{n-1}_i)$ has to be the left inverse of
    $\rho(\delta^n_i)=\delta^n_{n-i}$ and
    $\rho(\delta^n_{i-1})=\delta^n_{n+1-i}$, which is enough to reconstruct it
    thanks to the injectivity of the right inverses and determine that it is
    precisely $\sigma^{n-1}_{n-i-1}$. This is enough to prove that, if such an
    endofunctor exists, then it is unique since these arrows generate $\Delta$.

    One verifies that all of these associations preserve the desired relations
    and, since $\Delta$ is obtained by taking the free category generated by
    these arrows and then quotienting by the aforementioned equations, we get
    that $\rho$ does define an endofunctor $\Delta\rightarrow\Delta$, which one
    can verify to be an involution as it defines one on the morphisms
    generating the category. It follows that it also defines an involution
    $\rho^*\colon\sSet\rightarrow\sSet$. Also, notice that the functor $\rho$ is
    obtained simply by reversing the orderings of the elements of each $[n]$, so
    it acts on the simplices by ``inverting'' the faces.

    The isomorphism $\phi\colon N(\cC)^{\op}\rightarrow N(\cC^{\op})$ is given
    by sending $f\colon\Delta^1\rightarrow N(\cC)^{\op}$ to
    $\rho^*(f)^{\op}\colon\Delta^1\rightarrow N(\cC^{\op})$. Also, given a
    commutative triangle $(f,g,h)$ exhibited by a 2-simplex $t\rightarrow
    N(\cC)^{\op}$ in $N(\cC)^{\op}$, we see that applying
    $\rho^*$ turns it into another commutative triangle
    $(\rho^*(g),\rho^*(f),\rho^*(h))$ exhibited by $\rho^*(t)$. Looking at the
    description of $\rho^*$, we see that this actually corresponds to a
    commutative triangle in the category $\cC$ and it returns our starting
    triangle $(f,g,h)$ when we reapply $\rho^*$. But then, if
    $\rho^*(g)\cdot\rho^*(f)=\rho^*(h)$ in $\cC$, we get that
    $\rho^*(f)^{\op}\cdot\rho^*(g)^{\op}=\rho^*(h)^{\op}$ in $\cC^{\op}$.
    Similarly,
    $\rho^*(\id_x)^{\op}=\rho^*(s^0_0(x))^{\op}=s^0_0(\rho^*(x))^{\op}=\id_{\rho^*(x)}^{\op}=\id_x^{\op}$
    and therefore our natural transformation is well defined.
    We still have to check that it is an isomorphism. To do this we show that
    $N(\cC)^{\op}$ satisfies the Grothendieck-Segal condition and then we are
    done since the arrows are obtained by formally reversing the ones of
    $N(\cC)$, while our natural transformation is just reversing them twice and
    therefore it is essentially an identity on maps.
    \[
        \begin{tikzcd}
            \sSet(\Delta^n,N(\cC)^{\op})\ar[r]\ar[d, "\rho^*"]
            & \sSet(\Lambda_i^n,N(\cC)^{\op})\ar[d, "\rho^*"] \\
            \sSet(\Delta^n,N(\cC))\ar[r]
            & \sSet(\Lambda_{n-i}^n,N(\cC)) \\
        \end{tikzcd}
    \]
    The vertical arrows and the bottom one in this commutative diagram are
    isomorphisms for all $0<i<n$, hence the top one has to be an isomorphism
    too.

    (2) A similar proof applies to this case. Indeed, we may consider the
    commutative diagram
    \[
        \begin{tikzcd}
            \sSet(\Delta^n,X^{\op})\ar[r]\ar[d, "\rho^*"]
            & \sSet(\Lambda_i^n,X^{\op})\ar[d, "\rho^*"] \\
            \sSet(\Delta^n,X)\ar[r]
            & \sSet(\Lambda_{n-i}^n,X) \\
        \end{tikzcd},
    \]
    where the vertical arrows are isomorphisms and the bottom one is surjective
    for all $0<i<n$, which implies that the top one is surjective too.
\end{proof}

~\\
\exercise{3}
\begin{proof}


(i) It suffices to show that the functor $\Cat(-,\cC)$ is represented by $\cC^\simeq$ for each $\cC\in\Ob(\Cat)$. To this end, we note that for every $\cG\in\Gpd$, any functor $F\colon\cG\to\cC$ factorizes uniquely through $\cC^\simeq$, because $F(f)$ is an isomorphism for any (iso-)morphism $f$ in $\cG$, and if $F$ factorizes as 
\[
\cG\overset{F'}{\to}\cC\hookrightarrow\cC^\simeq\text{ and }\cG\overset{F''}{\to}\cC\hookrightarrow\cC^\simeq
\]
then $F'=F''$ on objects while for any morphism $f$ in $\cG$, $F'(f)=F(f)=F''(f)$ (so $F'=F''$). This gives a bijection
\[
\Cat(\cG,\cC)\cong\Gpd(\cG,\cC^\simeq).
\]
To see the functoriality, take any $G\colon\cG\to\cG'$ in $\Gpd$. Then we have a commutative diagram
\[
\begin{tikzcd}
\Cat(\cG',\cC)\arrow[r,equal,"\sim"]\arrow[d]& \Gpd(\cG',\cC^\simeq)\arrow[d]\\
\Cat(\cG,\cC)\arrow[r,equal,"\sim"]& \Gpd(\cG,\cC^\simeq)
\end{tikzcd}
\ 
\begin{tikzcd}[column sep=small]
F\arrow[rr,mapsto]\arrow[d,mapsto]&& F'\arrow[d,mapsto]\\
F\circ G\arrow[r,mapsto]& (F\circ G)'\arrow[r,equal]& F'\circ G
\end{tikzcd}
\]
where $F$, $F\circ G$ factorize through $F'$, $(F\circ G)'$ respectively. Note that $F'\circ G=(F\circ G)'$ since the composite $\cG\overset{\cG}{\to}\cG'\overset{F'}{\to}\cC^\simeq\hookrightarrow\cC$ is $F\circ G$.

(ii) We claim that subgroupoids of $EX$ are of the form
\[
\coprod_{i\in I}EX_i
\]
where $(X_i)_{i\in I}$ is a family of disjoint subsets of $X$. Indeed, such subcateories $\coprod_{i\in I}EX_i$ is a groupoid, and thus a subgroupoid of $X$. On the other hand, for any subgroupoid $Y$ of $X$, we define $I$ to be the set of isomorphism classes of objects in $Y$. Therefore $Y=\coprod_{i\in I}Ei$, which can be seen from the fact that $\Ob(Y)=\Ob(\coprod_{i\in I}Ei)$ and for any $x,y\in\Ob(Y)$, 
\[
Y(x,y)=\coprod_IEi(x,y)=\left\{\begin{array}{ll}
\varnothing& \text{ if $x,y$ are not isomorphic}\\
\{(x,y)\}&\text{ if $x,y$ are isomorphic}
\end{array}\right.
\]

(iii) It is enough to show that for all small set $X$, the functor $\Set(\Ob(-),X)$ is represented by $EX$. To this end, for any map $F\colon\Ob(\cC)\to X$, we define a functor $\widetilde{F}$ by letting
\begin{itemize}
\item $\widetilde{F}(x)=F(x)$ for any $x\in\Ob(\cC)$;

\item $\cC(x,y)\to EX(Fx,Fy)$ is the constant map, sending each morphism $f\colon x\to y$ to $(Fx,Fy)$.
\end{itemize}
and we get a bijection
\begin{align*}
\Set(\Ob(\cC),X)&\to\Cat(\cC,EX)\\
F&\mapsto \widetilde{F}\\
\Ob(F)&\mapsfrom F
\end{align*}
(the verification of them being mutually inverse is straightforward).

As for the functoriality, take any functor $G\colon\cC\to\cC'$. Then the diagram
\[
\begin{tikzcd}
\Set(\Ob(\cC'),X)\arrow[r,equal,"\sim"]\arrow[d]& \Cat(\cC',EX)\arrow[d]\\
\Set(\Ob(\cC),X)\arrow[r,equal,"\sim"]& \Cat(\cC,EX)
\end{tikzcd}
\ 
\begin{tikzcd}[column sep=small]
F\arrow[rr,mapsto]\arrow[d,mapsto]&& \widetilde{F}\arrow[d,mapsto]\\
F\circ\Ob(G)\arrow[r,mapsto]& \widetilde{F}\circ G\arrow[r,equal]& \widetilde{F\circ\Ob(G)}
\end{tikzcd}
\]
is commutative. Here $\widetilde{F}\circ G=\widetilde{F\circ\Ob(G)}$ because they both equal to $F\circ \Ob(G)$ on objects and hence they are the same on morphisms (since the map between hom sets $\cC(x,y)\to EX(F(G(x)),F(G(y))$ is the constant map).

(iv) Let us denote the functor sending $X$ to its associated discrete category by $\Disc$. We write $C\colon\cC\to\Set$ for the constant functor sending each $X\mapsto *$. We will show that the functor $\Cat(\cC,\Disc(-))$ is represented by $\pi_0(\cC)$ for all $\cC\in\Ob(\Cat)$. First of all, we define a map
\[
\Phi\colon\Set(\pi_0(\cC),S)\to\Cat(\cC,\Disc(S))
\]
by letting for every $F\colon\pi_0(\cC)\to S$ 
\begin{itemize}
\item $\Ob(\Phi(F))\colon\Ob(\cC)\to S$, $X\mapsto F\circ\iota_X(*)$, and

\item $\cC(X,Y)\to\Disc(S)(\Phi X,\Phi Y)$ be $\left\{\begin{array}{ll}
\varnothing,&\text{ if }\Phi X\neq\Phi Y\\
\{\id\},&\text{ if }\Phi X=\Phi Y,
\end{array}\right.$
\end{itemize}
where $\iota\colon C\to \pi_0(\cC)_\cC$ is the coprojection.
\[
\begin{tikzcd}
C(X)=*\arrow[dr,"\iota_X"]\arrow[drrr,bend left=15,"*\mapsto G(X)"]\arrow[dd,"\id_*"']&&&\\
&\colim_\cC C\arrow[rr,dotted,"\Psi(G)"]&& S\\
C(Y)=*\arrow[ur,"\iota_Y"']\arrow[urrr,bend right=15,"*\mapsto G(Y)"']&&&
\end{tikzcd}
\]
Next we intend to define an inverse $\Psi$ to $\Phi$. For any functor $G\colon\cC\to\Disc(S)$, note that $G(X)=G(Y)$ if there is a morphism $X\to Y$ in $\cC$. From this we get a cocone $C\to S_\cC$ with $C(X)\to S$ sending $*\mapsto G(X)$, which defines a unique map $\colim_\cC C\to S$ via the universal property of colimits and we denote it by $\Psi(G)$. 

To see that $\Psi$ and $\Phi$ are mutually inverse, we have
\[
\Phi\circ\Psi(G)(X)=\Psi(G)\circ\iota_X(*)=G(X)
\]
for all $X\in\Ob(\cC)$ and $G\colon\cC\to\Disc (S)$, and
\[
(\Psi\circ\Phi(F))\circ\iota_X(*)=(\Psi(X\mapsto F\circ\iota_X(*)))\circ\iota_X(*)=F\circ\iota_X(*)
\]
for all $X\in\Ob(\cC)$ and $F\colon\pi_0(\cC)\to S$. Therefore $\Psi\circ\Phi=\id$. Also, since the target of $\Phi\circ\Psi(G)$ is $\Disc(S)$, in which the hom sets are either $\varnothing$ or $\id$, we have $\Phi\circ\Psi=\id$. 

As for the functoriality, one has the following commutative diagram
\[
\begin{tikzcd}
\Set(\pi_0(\cC),S)\arrow[r,equal,"\sim"]\arrow[d]& \Cat(\cC,\Disc(S))\arrow[d]\\
\Set(\pi_0(\cC),S')\arrow[r,equal,"\sim"]& \Cat(\cC,\Disc(S'))
\end{tikzcd}
\ 
\begin{tikzcd}[column sep=small]
F\arrow[rr,mapsto]\arrow[d,mapsto]&& \Phi(F)\arrow[d,mapsto]\\
s\circ F\arrow[r,mapsto]& \Phi(s\circ F)\arrow[r,equal]& \Disc(s)\circ\Phi(F)
\end{tikzcd}
\]
for any map $s\colon S\to S'$ of sets. Here the equality is because
\[
\Phi(s\circ F)(X)=(s\circ F)\circ \iota_X(*)
\]
and
\[
\Disc(s)\circ\Phi(F)(X)=s\circ\Ob(\Phi(F))(X)=s\circ(F\circ\iota_X(*))
\]
for all $X\in\Ob(\cC)$.

(v) For a groupoid $\cG$, $\pi_0(\cG)$ is the set of isomorphism classes of $\cG$. This can be seen by verifying the universal property of colimits. For the moment we denote by $\pi'_0(\cG)$ the set of isomorphism classes. Define the coprojections $\iota_X\colon C(X)\to\pi_0'(\cG)$ by sending $*\mapsto[X]$ (the isomorphism class of $X\in\Ob(\cG)$). Suppose that we have a cocone $F\colon C\to S_\cG$ for some small set $S$. Then we can define a map
\[
f\colon\pi'_0(\cG)\to S
\]
by $[X]\mapsto F_X(*)$. This is well-defined, since $F_X=F_Y\circ\id_*$ whenever $X\cong Y$. Such $f$ is unique, since if there is another $f'\colon\pi_0'(\cG)\to S$, then 
\[
f'([X])=f'\circ \iota_X(*)=F_X(*)=f\circ\iota_X(*)=f([X])
\]
for all $X\in\Ob(\cG)$. This shows $\pi_0'(\cG)\cong\pi_0(\cG)$.
\end{proof}

\end{document}

\documentclass[a4paper,11pt,openany]{scrartcl}
\usepackage{basic}

\begin{document}

\title{Higher Category Theory\\ Assignment 2}

\author{Matteo Durante}

\maketitle

\exercise{1}
\begin{proof}
    We know from the description of colimits in $\Set$ that
    $\colim_{A\in\cF}A=(\bigsqcup_{A\in\cF}A)/\sim$, where $\sim$ is the
    equivalence relation generated by saying that $(a,A)\sim(b,B)$
    when there exist $f\colon A\rightarrow C$,
    $g\colon B\rightarrow C$ with $f(a)=g(b)$ (by $(x,X)$ we refer to
    $x$ seen as an element of $X$ in the disjoint union).

    Let's call $\sim_1$ the generating relation.
    By definition, if $f(a)=b$, picking $f$ and $\id_B$ we get $(a,A)\sim_1(b,B)$.

    We want to show that $(a,A)\sim(b,B)$ if and only if $a=b\in A\cap B$.

    Since our maps are inclusions and we have all intersections, we see that for
    any $A\cup B\subset C\in\cF$ the diagram
    \[
        \begin{tikzcd}
            A\cap B\ar[r, "i_{A\cap B,B}"]\ar[d, "i_{A\cap B,A}"]
            & B\ar[d, "i_{B,C}"] \\
            A\ar[r, "i_{A,C}"]
            & C
        \end{tikzcd}
    \]
    is a pullback square in $\cF$, which means that
    $(a,A)\sim_1(b,B)$ implies $a=b\in A\cap B$. Also, by our previous
    consideration, for any $a=b\in A\cap B$ we have that $(a,A)\sim_1(a,A\cap
    B)=(b,A\cap B)\sim_1(b,B)$ and therefore $(a,A)\sim(b,B)$.

    Now, calling $\sim_2$ the relation defined by writing
    $(a,A)\sim_2(b,B)$ when $a=b\in A\cap B$, we want to show that this is an
    equivalence relation, which with $\sim_1\subset\sim_2\subset\sim$ would
    imply that $\sim=\sim_2$.

    Reflexivity and simmetry are trivial, so let us check transitivity. Suppose
    $(a,A)\sim_2(b,B)\sim_2(c,C)$. By applying our definition twice, we get that
    $a=b=c\in A\cap B\cap C\subset A\cap C$, which is what we needed.

    Consider now the canonical map
    $\pi\colon\bigsqcup_{A\in\cF}A\rightarrow\bigcup_{A\in\cF}A$, $(a,A)\mapsto
    a$, which is
    obviously surjective. We want to show that two
    elements are mapped to the same one if and only if they satisfy the
    aforementioned relation, so suppose that $\pi(a,A)=\pi(b,B)$.
    This is equivalent to saying that $a=b\in A\cap B\in\cF$ and therefore, by our previous
    observations, to $(a,A)\sim(b,B)$.
\end{proof}

~\\
\exercise{2}
\begin{proof}
    Let us consider a functor $F\colon\cA\times\cB\rightarrow\cC$ and define
    $\phi(F)\colon\cA\rightarrow\underline{\Cat}(\cB,\cC)$ on objects as
    $\phi(F)(a)(b)=F(a,-)(b)=F(a,b)$, $\phi(F)(a)(f)=F(\id_a,f)$.

    $\phi(F)(a)$ is indeed a functor because
    $\phi(F)(a)(\id_b)=F(\id_a,\id_b)=F(\id_{(a,b)})=\id_{F(a,b)}=\id_{\phi(F)(a)(b)}$ and
    $\phi(F)(a)(g\cdot f)=F(\id_a,g\cdot f)=F(\id_a,g)\cdot
    F(\id_a,f)=\phi(F)(a)(g)\cdot\phi(F)(a)(f)$.

    Let us set $\phi(F)(f)(b)=F(f,\id_b)$, for $f\colon
    a\rightarrow a'$ in $\cA$. We have to check
    that $\phi(F)(f)$ is a natural transformation
    $\phi(F)(a)\Rightarrow\phi(F)(a')$. For this we consider a morphism $g\colon
    b\rightarrow b'$ in $\cB$ and observe that
    $\phi(F)(a')(g)\cdot\phi(F)(f)(b)=F(\id_{a'},g)\cdot
    F(f,\id_b)=F(f,g)=F(f,\id_{b'})\cdot
    F(\id_a,g)=\phi(F)(f)(b')\cdot\phi(F)(a)(g)$, which proves our claim.

    The assignment is functorial because
    $\phi(F)(\id_a)(b)=F(\id_a,\id_b)=\id_{F(a,b)}=\id_{\phi(F)(a)(b)}$, which implies that
    $\phi(F)(\id_a)=\id_{\phi(F)(a)}$, and
    $\phi(F)(g\cdot f)(b)=F(g\cdot f,\id_b)=F(g,\id_b)\cdot
    F(f,\id_b)=\phi(F)(g)(b)\cdot\phi(F)(f)(b)$ for every $b\in\cB$, thus
    $\phi(F)(g\cdot f)=\phi(F)(g)\cdot\phi(F)(f)$.

    Now we consider a natural transformation $\alpha\colon F\Rightarrow G$ and
    define $\phi(\alpha)(a)$ by associating $b\in\cB$ to $\alpha_{(a,b)}$. The
    assignment clearly satisfies
    $\phi(\beta\cdot\alpha)(a)(b)=(\beta\cdot\alpha)_{(a,b)}=\beta_{(a,b)}\cdot\alpha_{(a,b)}=\phi(\beta)(a)(b)\cdot\phi(\alpha)(a)(b)$,
    hence $\phi(\beta\cdot\alpha)=\phi(\beta)\cdot\phi(\alpha)$ and we only need to
    check that $\phi(\alpha)(a)$ is indeed a natural transformation, which
    follows from the fact that
    $\phi(\alpha)(a)(b')\cdot\phi(F)(a)(g)=\alpha_{(a,b')}\cdot
    F(\id_a,g)=G(\id_a,g)\cdot\alpha_{(a,b)}=\phi(G)(a)(g)\cdot\phi(\alpha)(a)(b)$.

        We have just proved that $\phi$ is a functor, so we only have to show
        that it is an isomorphism, for which we will construct an inverse
        $\psi$.

    Given a functor $F\colon\cA\rightarrow\underline{\Cat}(\cB,\cC)$, we set
    $\psi(F)(a,b)=F(a)(b)$ and, given $f\colon a\rightarrow a'$, $g\colon
    b\rightarrow b'$, $\psi(F)(f,g)=F(f)(b')\cdot F(a)(g)$. Notice that, by
    naturality of $F(f)\colon F(a)\Rightarrow F(a')$, we have
    $\psi(F)(f,g)=F(a')(g)\cdot F(f)(b)$.

    This is indeed a functor because
    \begin{align*}
        \psi(F)(\id_{(a,b)}) &=\psi(F)(\id_a,\id_b) \\
        &=F(\id_a)(b)\cdot F(a)(\id_b) \\
        &=\id_{F(a)}(b)\cdot\id_{F(a)(b)} \\
        &=\id_{F(a)(b)} \\
        &=\id_{\psi(F)(a,b)}
    \end{align*}
    and
    \begin{align*}
        \psi(F)((f',g')\cdot(f,g)) &=\psi(F)(f'f,g'g) \\
        &=F(f'f)(b'')\cdot F(a)(g'g) \\
        &=F(f')(b'')\cdot F(f)(b'')\cdot F(a)(g')\cdot F(a)(g) \\
        &=F(f')(b'')\cdot F(a')(g')\cdot F(f)(b')\cdot F(a)(g) \\
        &=\psi(F)(f',g')\cdot\psi(F)(f,g)
    \end{align*}

    Now, given a natural transformation $\alpha\colon F\Rightarrow G$, we define
    $\psi(\alpha)$ by setting $\psi(\alpha)(a,b)=\alpha(a)(b)$. This is a
    natural transformation $\psi(F)\Rightarrow\psi(G)$ since
    \begin{align*}
        \psi(G)(f,g)\cdot\psi(\alpha)(a,b) &=G(f)(b')\cdot
        G(a)(g)\cdot\alpha(a)(b) \\
        &=G(f)(b')\cdot\alpha(a)(b')\cdot F(a)(g) \\
        &=\alpha(a')(b')\cdot F(f)(b')\cdot F(a)(g) \\
        &=\psi(\alpha)(a',b')\cdot\psi(F)(f,g)
    \end{align*}

    We now check that the associations $\phi$ and $\psi$ are inverse to each
    other.

    Let's start by considering functors $F,G\colon\cA\times\cB\rightarrow\cC$,
    a morphism $(f,g)\colon(a,b)\rightarrow(a',b')$ and a natural transformation
    $\alpha\colon F\Rightarrow G$.
    \begin{align*}
        \psi(\phi(F))(a,b) &=\phi(F)(a)(b) \\
        &=F(a,b) \\
        \psi(\phi(F))(f,g) &=\phi(F)(f)(b')\cdot\phi(F)(a)(g) \\
        &=F(f,\id_{b'})\cdot F(\id_a,g) \\
        &=F(f\cdot\id_a,\id_{b'}\cdot g) \\
        &=F(f,g) \\
        \psi(\phi(\alpha))(a,b) &=\phi(\alpha)(a)(b) \\
        &=\alpha(a,b)
    \end{align*}
    It follows that $\psi\circ\phi=\id$.

    Similarly, let's consider functors $F,G\colon\cA\rightarrow\underline{\Cat}(\cB,\cC)$,
    morphisms $f\colon a\rightarrow a'$, $g\colon b\rightarrow b'$ and a natural
    transformation $\alpha\colon F\Rightarrow G$.
    \begin{align*}
        \phi(\psi(F))(a)(b) &=\psi(F)(a,b) \\
        &=F(a)(b) \\
        \phi(\psi(F))(f)(b) &=\psi(F)(f,\id_b) \\
        &=F(f)(b)\cdot F(a)(\id_b) \\
        &=F(f)(b)\cdot\id_{F(a)(b)} \\
        &=F(f)(b) \\
        \phi(\psi(F))(a)(g) &=\psi(F)(\id_a,g) \\
        &=F(\id_a)(b')\cdot F(a)(g) \\
        &=\id_{F(a)}(b')\cdot F(a)(g) \\
        &=\id_{F(a)(b')}\cdot F(a)(g) \\
        &=F(a)(g) \\
        \phi(\psi(\alpha))(a,b) &=\psi(\alpha)(a)(b) \\
        &=\alpha(a,b)
    \end{align*}
    This implies that $\phi\circ\psi=\id$, which concludes he proof.
\end{proof}

~\\
\exercise{3}
\begin{proof}
    We will prove the first three statements in the general case of an adjunction
    $F\dashv G$, $F\colon\cC\rightarrow\cD$.

    (1) First of all, for any object $c\in\cC$, the identity $\id_{Fc}\in\cD(Fc,Fc)$
    is associated under the natural isomorphism $\cC(c,GFc)\cong\cD(Fc,Fc)$ to a
    unique map $\eta_c\colon c\rightarrow GFc$. We want to show that the
    collection $(\eta_c)_{c\in\cC}$ defines a natural transformation
    $\id_\cC\Rightarrow GF$. In order to do this, we consider a map
    $f\colon c\rightarrow c'$ in $\cC$ and notice that the commutativity of the
    diagram
    \[
        \begin{tikzcd}
            \cD(Fc',Fc')\ar[r, "\sim"]\ar[d, "Ff^*"]
            & \cC(c',GFc')\ar[d, "f^*"] \\
            \cD(Fc,Fc')\ar[r, "\sim"]
            & \cC(c,GFc')
        \end{tikzcd}
    \]
    given by the natural isomorphism implies that
    $\widehat{g\cdot Ff}=\widehat{Ff^*(g)}=f^*(\hat{g})=\hat{g}\cdot f$.
    Similarly, working by post-composing and considering the corresponding
    commutative square induced by the adjunction, we get that
    $\widehat{Ff\cdot h}=\widehat{Ff_*(h)}=GFf_*(\hat{h})=GFf\cdot\hat{h}$.

    Remembering our definition of $\eta$, we see that $\eta_{c'}\cdot
    f=\widehat{\id_{c'}\cdot
    Ff}=\widehat{Ff}=\widehat{Ff\cdot\id_c}=GFf\cdot\eta_c$, which proves our
    claim.

    We will still prove that a commutative square
    \[
        \begin{tikzcd}
            Fc\ar[r, "f"]\ar[dr, "k"]\ar[d, "Fg"]
            & Fd\ar[d, "Fg'"] \\
            Fc'\ar[r, "f'"]
            & Fd'
        \end{tikzcd}
    \]
    induces another one
    \[
        \begin{tikzcd}
            c\ar[r, "\hat{f}"]\ar[dr, "\hat{k}"]\ar[d, "g"]
            & GFd\ar[d, "GFg'"] \\
            c'\ar[r, "\hat{f'}"]
            & GFd'
        \end{tikzcd},
    \]
    but this follows from the equalities we have exibited earlier appllied to
    the individual triangles.

    (2) Consider a functor $D\colon\cI\rightarrow\cC$ admitting a
    universal cocone $\alpha\colon D\rightarrow\colim_{\cI}D$
    and such that for every
    $i\in\cI$ the morphism $\eta_{Di}$ is an isomorphism. Composing $D$ with
    $F$ gives us a diagram $FD\colon\cI\rightarrow\cD$ and we may
    consider the natural transformation $\id_{FD}$ and, since left adjoints
    preserve colimits, the universal cocone
    $F\alpha\colon FD\Rightarrow F\colim_{\cI}D$. This
    gives us a commutative diagram
    \[
        \begin{tikzcd}
            FD\ar[d, "F\alpha"]\ar[r, "\id_{FD}"]
            & FD\ar[d, "F\alpha"] \\
            F\colim_{\cI}D\ar[r, "\id_{F\colim_{\cI}D}"]
            & F\colim_{\cI}D
        \end{tikzcd}.
    \]
    Now, applying the natural isomorphism given by the adjunction to the upper
    triangle and the lower one, we get a new commutative square by substituting
    the individual morphisms of the horizontal natural transformations by their
    transpositions as shown in (2) (there we were working with individual
    morphisms, but the principle is the same as the same diagrams show that
    these substitutions return new natural transformations).
    \[
        \begin{tikzcd}
            D\ar[d, "\alpha"]\ar[r, "\eta_D"]
            & GFD\ar[d, "GF\alpha"] \\
            \colim_{\cI}D\ar[r, "\eta_{\colim_{\cI}D}"]
            & GF\colim_{\cI}D
        \end{tikzcd}
    \]
    Observe that the upper arrow is an isomorphism between the diagrams and
    therefore the factorization of the induced cocone $D\rightarrow
    GF\colim_\cI D$ is itself an isomorphism, but by construction this is
    precisely $\eta_{\colim_{\cI}D}$.

    (3) Consider the diagram
    \[
        \begin{tikzcd}
            \cC(c,d)\ar[r, "(\eta_Y)_*"]\ar[dr, swap, "F"]
            & \cC(c,GFd)\ar[d, "\sim"] \\
            & \cD(Fc,Fd)
        \end{tikzcd}.
    \]
    It commutes because, as shown earlier, $(\eta_d\cdot
    f)^\#=(\eta_d)^\#\cdot Ff=\id_Y\cdot Ff=Ff$ (notice that here we
    are using the equality found in (1) but with the inverse of $\hat{(-)}$,
    which we denote by $(-)^\#$). The vertical map is an isomorphism, so the
    diagonal map is too if and only if $\eta_Y$ is.

    (4) We may suppose the categories $\cA,\cB$ to be small by choosing a large
    enough universe.

    If $u_!$ is fully faithful, then consider the diagram
    \[
        \begin{tikzcd}
            \cA\ar[r, "\yo"]\ar[d, "u"]
            & \hat{\cA}\ar[d, "u_!"] \\
            \cB\ar[r, "\yo"]
            & \hat{\cB}
        \end{tikzcd}.
    \]
    Remembering that $u_!$ is defined by $u_!(X)=\colim_{\yo/X}(\yo\circ u)(a)$,
    we have that $u_!(\yo_a)=\colim_{\yo/\yo_a}\yo_{u(a)}$, which is precisely
    $\yo_{u(a)}$ because the indexing category has $\id_{\yo_a}$ as a terminal
    object. It follows that the diagram commutes and, since the horizontal
    arrows and the one on the right are fully faithful, so is the one on the
    left.

    Viceversa, suppose that $u$ is fully faithful. We know that
    $u_!\circ\yo=\yo\circ u$ is fully faithful, which by the considerations in
    (2) implies that $\eta_X$ is
    an isomorphism whenever $X$ is a representable presheaf on $\cA$.
    By (3), the class of presheaves for which $\eta_X$ is an isomorphism is
    closed under small colimits and, since the representable ones are dense in
    $\hat{\cA}$, this implies that it contains all presheaves. We now apply (2).
\end{proof}

~\\
\exercise{4}
\begin{proof}
    (a) Assuming to work in a large enough universe, we can suppose the
    categories $\cC,\cD$ to be locally small. Suppose to have a diagram $D\colon\cI\rightarrow\cC$ admitting a
    colimit. Using the natural isomorphism in both variables given by the
    adjunction and the fact that the contravariant hom functor preserves
    colimits by sending them to limits, we have the following chain of natural
    isomorphisms:
    \begin{align*}
        \cD(F\colim_\cI D,Y) &\cong \cC(\colim_\cI D,GY) \\
        &\cong\lim_{\cI^{\op}}\cC(D,GY) \\
        &\cong\lim_{\cI^{\op}}\cD(FD,Y) \\
        &\cong\cD(\colim_{\cI}FD,Y).
    \end{align*}
    By Yoneda, it follows that $F\colim_\cI D\cong\colim_\cI FD$.

    (b) Again we consider a large enough universe and a diagram
    $D\colon\cI\rightarrow\cD$ admitting a limit. Similarly to before, making
    use of the preservation of limits by the hom functor, we have the following
    chain of natural isomorphisms:
    \begin{align*}
        \cC(X,G\lim_\cI D) &\cong \cD(FX,\lim_\cI D) \\
        &\cong\lim_{\cI}\cD(FX,D) \\
        &\cong\lim_{\cI}\cC(X,GD) \\
        &\cong\cC(X,\lim_\cI GD).
    \end{align*}
    Again by Yoneda, we get $G\lim_\cI D\cong\lim_\cI GD$.

    (c) It is enough to show that $\cC/Y$ is isomorphic to $\cC/GY$, which has a
    final object given by $\id_{GY}$. Our isomorphism will be induced by the
    natural isomorphism $\alpha\colon\cD(F,Y)\xrightarrow{\sim}\cC(-,GY)$ and
    will send $g\colon FX\rightarrow Y$ to $\hat{g}\colon X\rightarrow GY$,
    $f\colon h\rightarrow k$ to $f\colon\hat{h}\rightarrow\hat{k}$.

    This is well-defined because $\alpha$ is a natural isomorphism, which
    implies that the triangles
    \[
        \begin{tikzcd}
            FX\ar[d, "Ff", swap]\ar[dr, "g"] \\
            FX'\ar[r, "h", swap]
            & Y
        \end{tikzcd}
        \quad
        \begin{tikzcd}
            X\ar[d, "f", swap]\ar[dr, "\hat{g}"] \\
            X'\ar[r, "\hat{h}", swap]
            & GY
        \end{tikzcd}
    \]
    commute if and only if the other one commutes.

    It is a functor because it trivially preserves identities and
    compositions, while faithfullness is trivial. The bijection on objects
    follows from the fact that
    a map $g\in\cC(X,GY)$ is the image of a unique map $h\in\cD(FX,Y)$ under
    $\alpha$ and fullness is just a consequence of the condition of
    commutativity of the previous triangles, which means that every map
    $f\colon\hat{g}\rightarrow\hat{h}$ in $\cC/GY$ is the image of the ``same''
    map $f\colon g\rightarrow h$ in $\cC/Y$.
\end{proof}

\end{document}

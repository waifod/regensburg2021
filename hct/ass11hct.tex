\documentclass[a4paper,11pt,openany]{scrartcl}
\usepackage{../reg2021}
\usepackage{../quiver}

\begin{document}
\noindent\textbf{Author}\hfill\textbf{Semester} \linebreak
\vspace*{-.1cm} Matteo Durante, Yuhao Zhang\hfill WS 2020/21 \\

\noindent
\rule{\linewidth}{1pt}
\begin{center}
\Large
\textbf{Higher Category Theory} \\
Assignment 11
\end{center}
\rule{\linewidth}{1pt}
\\

%%%%%%%%%%%%%%%%%%%%%%%%%%%%%%%%%%%%%%%%%%%%%%%%%%%%%%%%%%%%%%%%%%%%%%%%%%%%%%%

%%%sheet4
\newcommand{\La}{\Lambda}
\newcommand{\pa}{\partial}
\newcommand{\ob}{\operatorname{Ob}}
\newcommand{\mor}{\operatorname{Mor}}
\newcommand{\sto}{\twoheadrightarrow}

%%%sheet5
\newcommand{\plim}{\varprojlim}
\newcommand{\sst}{\subseteq}
\newcommand{\eq}{\operatorname{eq}}

%%%sheet6
\newcommand{\f}{\varphi}

%%%sheet8
\newcommand{\sing}{\operatorname{Sing}}

%%%sheet9
\newcommand{\ihom}{\underline{\Hom}}

~\\
\exercise{1}
\begin{proof}
    We start by drawing the commutative cube in question.
    \[\begin{tikzcd}
        & {B_1} \\
        {A_1} &&& {B_1\sqcup_{A_1}C_1} \\
        && {C_1} \\
        & {B_2} \\
        {A_2} &&& {B_2\sqcup_{A_2}C_2} \\
        && {C_2}
        \arrow["h"{pos=0.3}, from=3-3, to=6-3, crossing over]
        \arrow["{a_2}"', from=5-1, to=6-3]
        \arrow[from=6-3, to=5-4]
        \arrow["{i_2}"', from=5-1, to=4-2]
        \arrow[from=4-2, to=5-4]
        \arrow["k", from=2-4, to=5-4]
        \arrow["f"', from=2-1, to=5-1]
        \arrow["{a_1}"'{pos=0.2}, from=2-1, to=3-3, crossing over]
        \arrow[from=1-2, to=2-4]
        \arrow[from=3-3, to=2-4]
        \arrow["{i_1}", from=2-1, to=1-2]
        \arrow["g"'{pos=0.7}, from=1-2, to=4-2]
    \end{tikzcd}\]
    Next we apply the functor $\Fun(-,W)$, where $W$ is a Kan complex.
    \[\begin{tikzcd}
        & {\Fun(B_2,W)} \\
        {\Fun(B_2\sqcup_{A_2}C_2,W)} &&& {\Fun(A_2,W)} \\
        && {\Fun(C_2,W)} \\
        & {\Fun(B_1,W)} \\
        {\Fun(B_1\sqcup_{A_1}C_1,W)} &&& {\Fun(A_1,W)} \\
        && {\Fun(C_1,W)}
        \arrow["{\Fun(h,W)}"{pos=0.3}, from=3-3, to=6-3, crossing over]
        \arrow[from=5-1, to=6-3]
        \arrow["{\Fun(i_1,W)}"', from=6-3, to=5-4]
        \arrow[from=5-1, to=4-2]
        \arrow["{\Fun(a_1,W)}"'{pos=0.3}, from=4-2, to=5-4]
        \arrow["{\Fun(f,W)}", from=2-4, to=5-4]
        \arrow["{\Fun(k,W)}"', from=2-1, to=5-1]
        \arrow[from=2-1, to=3-3, crossing over]
        \arrow["{\Fun(a_2,W)}", from=1-2, to=2-4]
        \arrow["{\Fun(i_2,W)}", from=3-3, to=2-4]
        \arrow[from=2-1, to=1-2]
        \arrow["{\Fun(g,W)}"'{pos=0.7}, from=1-2, to=4-2]
    \end{tikzcd}\]
    We know that a morphism is a weak homotopy equivalence if and only if its
    image under $\Fun(-,W)$ is a homotopy equivalence for any Kan complex
    $W$. Also, monomorphisms are mapped to Kan fibrations and, for any
    simplicial set $X$, the simplicial set $\Fun(X,W)$ is
    itself a Kan complex. Finally, $\Fun(-,W)$ preserves colimits by sending
    them to limits due to its contravariance because
    \begin{align*}
        \sSet(X,\Fun(\colim_{\cI}D_i,W))
        &\cong\sSet(X\times\colim_{\cI}D_i,W) \\
        &\cong\sSet(\colim_{\cI} X\times D_i,W) \\
        &\cong\lim_{\cI^{\op}}\sSet(X\times D_i,W) \\
        &\cong\lim_{\cI^{\op}}\sSet(X,\Fun(D_i,W)) \\
        &\cong\sSet(X,\lim_{\cI^{\op}}\Fun(D_i,W))
    \end{align*}
    naturally in $X$,
    thus the top and bottom squares are pullbacks and, being Kan fibrations
    saturated, all of the horizontal maps belong to their class.

    From our observations it follows that the top and bottom squares are
    pullbacks, the vertical morphisms (except for possibly the left one) in our
    latest diagram are homotopy equivalences, the horizontal maps are Kan
    fibrations and every object is a Kan complex.

    We can now apply a proposition from lecture 20 and conclude that $\Fun(k,W)$
    is itself a homotopy equivalence for any $W$, hence $k$ is a weak homotopy
    equivalence.
\end{proof}

~\\
\exercise{4}
\begin{proof}
    Consider a filtered diagram $D\colon\cI\rightarrow\sSet$.
    Since $\Lambda^n_k$ is a finite simplicial set, the functor
    $\sSet(\Lambda^n_k,-)$ preserves
    filtered colimits. It follows that, fixed a morphism
    $\alpha\colon\Delta^n_k\rightarrow\colim_{\cI}D_i$, we have an element
    $[\alpha_i]\in\colim_{\cI}\sSet(\Lambda^n_k,D_i)\cong\sSet(\Lambda^n_k,\colim_{\cI}D_i)$
    corresponding to it. This means that there is a $i\in\cI$ with a morphism
    $\alpha_i\colon\Lambda^n_k\rightarrow D_i$ such that
    \[\begin{tikzcd}
        {\Lambda^n_k} & {D_i} \\
        {} & {\colim_{\cI}D_i}
        \arrow["{\lambda_i}", from=1-2, to=2-2]
        \arrow["{\alpha_i}", from=1-1, to=1-2]
        \arrow["\alpha"', from=1-1, to=2-2]
    \end{tikzcd}\]
    commutes, where $\lambda_i$ is a leg of the cocone.

    Now, if the simplicial set $D_i$ is a Kan complex (or a
    $\infty$-category), the horn admits a
    filling $t\colon\Delta^n\rightarrow D_i$ for $0\leq k\leq n$ (respectively
    $0<k<n$), which gives us the commutative diagram
    \[\begin{tikzcd}
        {\Lambda^n_k} & {D_i} \\
        {\Delta^n} & {\colim_{\cI}D_i}
        \arrow["{\lambda_i}", from=1-2, to=2-2]
        \arrow["{\alpha_i}", from=1-1, to=1-2]
        \arrow["\alpha"{description, pos=0.25}, from=1-1, to=2-2]
        \arrow["t"', dotted, from=2-1, to=2-2]
        \arrow[from=1-1, to=2-1]
        \arrow["{t_i}"{description, pos=0.25}, dotted, from=2-1, to=1-2]
    \end{tikzcd}\]
    and in particular the $n$-simplex $t=\lambda_i\cdot t_i$ of
    $\colim_{\cI}D_i$ such that $t|_{\Lambda^n_k}=\lambda_i\cdot
    t_i|_{\Lambda^n_k}=\lambda_i\cdot\alpha_i=\alpha$.

    Now, if for every $i\in\cI$ the simplicial set $D_i$ is a Kan complex, this
    filling is always possible and therefore the colimit is itself a Kan
    complex. Similarly, if the simplicial sets in the diagram are
    $\infty$-categories the same goes for $\colim_{\cI}D_i$.
\end{proof}

\end{document}

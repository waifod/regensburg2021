\documentclass[a4paper,11pt,openany]{scrartcl}
\usepackage{../reg2021}
\usepackage{../quiver}

\begin{document}
\noindent\textbf{Author}\hfill\textbf{Semester} \linebreak
\vspace*{-.1cm} Matteo Durante, Yuhao Zhang\hfill WS 2020/21 \\

\noindent
\rule{\linewidth}{1pt}
\begin{center}
\Large
\textbf{Higher Category Theory} \\
Assignment 11
\end{center}
\rule{\linewidth}{1pt}
\\

%%%%%%%%%%%%%%%%%%%%%%%%%%%%%%%%%%%%%%%%%%%%%%%%%%%%%%%%%%%%%%%%%%%%%%%%%%%%%%%

%%%sheet4
\newcommand{\La}{\Lambda}
\newcommand{\pa}{\partial}
\newcommand{\ob}{\operatorname{Ob}}
\newcommand{\mor}{\operatorname{Mor}}
\newcommand{\sto}{\twoheadrightarrow}

%%%sheet5
\newcommand{\plim}{\varprojlim}
\newcommand{\sst}{\subseteq}
\newcommand{\eq}{\operatorname{eq}}

%%%sheet6
\newcommand{\f}{\varphi}

%%%sheet8
\newcommand{\sing}{\operatorname{Sing}}

%%%sheet9
\newcommand{\ihom}{\underline{\Hom}}

%%%sheet11
\newcommand{\N}{\mathbb{N}}


~\\
\exercise{1}
\begin{proof}
    We start by drawing the commutative cube in question.
    \[\begin{tikzcd}
        & {B_1} \\
        {A_1} &&& {B_1\sqcup_{A_1}C_1} \\
        && {C_1} \\
        & {B_2} \\
        {A_2} &&& {B_2\sqcup_{A_2}C_2} \\
        && {C_2}
        \arrow["h"{pos=0.3}, from=3-3, to=6-3, crossing over]
        \arrow["{a_2}"', from=5-1, to=6-3]
        \arrow[from=6-3, to=5-4]
        \arrow["{i_2}"', from=5-1, to=4-2]
        \arrow[from=4-2, to=5-4]
        \arrow["k", from=2-4, to=5-4]
        \arrow["f"', from=2-1, to=5-1]
        \arrow["{a_1}"'{pos=0.2}, from=2-1, to=3-3, crossing over]
        \arrow[from=1-2, to=2-4]
        \arrow[from=3-3, to=2-4]
        \arrow["{i_1}", from=2-1, to=1-2]
        \arrow["g"'{pos=0.7}, from=1-2, to=4-2]
    \end{tikzcd}\]
    Since monomorphisms are saturated, we know that all of the horizontal maps
    are monomorphisms (misread the assignment: I thought the maps $a_\epsilon$
    were monomorphisms too).

    Next we apply the functor $\Fun(-,W)$, where $W$ is a Kan
    complex.
    \[\begin{tikzcd}
        & {\Fun(B_2,W)} \\
        {\Fun(B_2\sqcup_{A_2}C_2,W)} &&& {\Fun(A_2,W)} \\
        && {\Fun(C_2,W)} \\
        & {\Fun(B_1,W)} \\
        {\Fun(B_1\sqcup_{A_1}C_1,W)} &&& {\Fun(A_1,W)} \\
        && {\Fun(C_1,W)}
        \arrow["{\Fun(h,W)}"{pos=0.3}, from=3-3, to=6-3, crossing over]
        \arrow[from=5-1, to=6-3]
        \arrow["{\Fun(i_1,W)}"', from=6-3, to=5-4]
        \arrow[from=5-1, to=4-2]
        \arrow["{\Fun(a_1,W)}"'{pos=0.3}, from=4-2, to=5-4]
        \arrow["{\Fun(f,W)}", from=2-4, to=5-4]
        \arrow["{\Fun(k,W)}"', from=2-1, to=5-1]
        \arrow[from=2-1, to=3-3, crossing over]
        \arrow["{\Fun(a_2,W)}", from=1-2, to=2-4]
        \arrow["{\Fun(i_2,W)}", from=3-3, to=2-4]
        \arrow[from=2-1, to=1-2]
        \arrow["{\Fun(g,W)}"'{pos=0.7}, from=1-2, to=4-2]
    \end{tikzcd}\]
    We know that a morphism is a weak homotopy equivalence if and only if its
    image under $\Fun(-,W)$ is a homotopy equivalence for any Kan complex
    $W$. Also, monomorphisms are mapped to Kan fibrations and, for any
    simplicial set $X$, the simplicial set $\Fun(X,W)$ is itself a Kan complex.
    Finally, $\Fun(-,W)$ preserves colimits by sending them to limits because
    \begin{align*}
        \sSet(X,\Fun(\colim_{\cI}D_i,W))
        &\cong\sSet(X\times\colim_{\cI}D_i,W) \\
        &\cong\sSet(\colim_{\cI} X\times D_i,W) \\
        &\cong\lim_{\cI^{\op}}\sSet(X\times D_i,W) \\
        &\cong\lim_{\cI^{\op}}\sSet(X,\Fun(D_i,W)) \\
        &\cong\sSet(X,\lim_{\cI^{\op}}\Fun(D_i,W))
    \end{align*}
    naturally in $X$, thus the top and bottom squares are pullbacks.

    Let's summarize what we have so far with respect to our latest diagram: the
    top and bottom squares are pullbacks, the vertical morphisms (except for
    possibly the left one) are homotopy equivalences, the horizontal maps are
    Kan fibrations and every object is a Kan complex.

    We can now apply a proposition from Lecture 20 and conclude that $\Fun(k,W)$
    is itself a homotopy equivalence for any $W$, hence $k$ is a weak homotopy
    equivalence.
\end{proof}

~\\
\exercise{2}
\begin{proof}
Applying $\Fun(-,W)$ to the diagram with $W$ an arbitrary Kan complex, we get a commutative diagram
\[
\begin{tikzcd}
\cdots\arrow[r]& \Fun(B_{n+1},W)\arrow[r,"j_{n+1}^*"]\arrow[d,"f_{n+1}^*"]& \Fun(B_n,W)\arrow[r]\arrow[d,"f_n^*"]& \cdots\arrow[r]& \Fun(B_1,W)\arrow[r,"j_1^*"]\arrow[d,"f_1^*"]& \Fun(B_0,W)\arrow[d,"f_0^*"]\\
\cdots\arrow[r]& \Fun(A_{n+1},W)\arrow[r,"i_{n+1}^*"]& \Fun(A_n,W)\arrow[r]& \cdots\arrow[r]& \Fun(A_1,W)\arrow[r,"i_1^*"]& \Fun(A_0,W)
\end{tikzcd}
\]
where $\Fun(A_n,W)$, $\Fun(B_n,W)$ are Kan complexes and every $i_n^*$, $j_n^*$ are Kan fibrations for all $n\geqslant0$ (Lecture 9). Since $f_n$ are weak homotopy equivalences ($n\geqslant0$), one has $f_n^*$ being homotopy equivalences as well (Lecture 18). Hence by a proposition in Lecture 20, it follows that $\lim_{\N^{\op}}\Fun(f_n,W)$ is a homotopy equivalence. From the proof of Exercise 1, we have $\lim_{\N^{\op}}\Fun(f_n,W)\cong\Fun(\colim_{\N}f_n,W)$. Therefore $f_\infty=\colim_{\N}f_n\colon A_\infty\to B_\infty$ is a weak homotopy equivalence.
\end{proof}

~\\
\exercise{3}
\begin{proof}
We construct the following commutative diagram
\[
\begin{tikzcd}
C_0\arrow[d,"h'"]& A_1\arrow[r,hookrightarrow,"i_1"]\arrow[d,"\id"]\arrow[l,hookrightarrow,"a_0"']& B_1\arrow[d,"\id"]\\
C_1\arrow[d,"h"]& A_1\arrow[r,hookrightarrow,"i_1"]\arrow[d,"f"]\arrow[l,"a_1"']& B_1\arrow[d,"g"]\\
C_2& A_2\arrow[r,"i_2"]\arrow[l,hookrightarrow,"a_2"']& B_2
\end{tikzcd}
\]
where the morphism $a_1\colon A_1\to C_1$ factorizes into $h'\cdot a_0$ with
    $a_0$ a monomorphism and $h'$ a trivial fibration. Recall that a trivial
    fibration is an absolute weak equivalence. Denote by $D_0$ the pushout of
    $a_0$ along $i_1$. We apply Exercise 1 to the first two rows and get $D_0\to
    D_1$ a weak homotopy equivalence. Also, applying Exercise 1 to the outer
    diagram yields that $D_0\to D_2$ is a weak homotopy equivalence. Therefore
    $D_1\to D_2$ is a weak homotopy equivalence.
\end{proof}

~\\
\exercise{4}
\begin{proof}
    Consider a filtered diagram $D\colon\cI\rightarrow\sSet$.
    Since $\Lambda^n_k$ is a finite simplicial set, the functor
    $\sSet(\Lambda^n_k,-)$ preserves
    filtered colimits. It follows that, fixed a morphism
    $\alpha\colon\Lambda^n_k\rightarrow\colim_{\cI}D_i$, we have an element
    $[\alpha_i]\in\colim_{\cI}\sSet(\Lambda^n_k,D_i)\cong\sSet(\Lambda^n_k,\colim_{\cI}D_i)$
    corresponding to it. This means that there is a $i\in\cI$ with a morphism
    $\alpha_i\colon\Lambda^n_k\rightarrow D_i$ such that
    \[\begin{tikzcd}
        {\Lambda^n_k} & {D_i} \\
        {} & {\colim_{\cI}D_i}
        \arrow["{\lambda_i}", from=1-2, to=2-2]
        \arrow["{\alpha_i}", from=1-1, to=1-2]
        \arrow["\alpha"', from=1-1, to=2-2]
    \end{tikzcd}\]
    commutes, where $\lambda_i$ is a leg of the cocone.

    Now, if the simplicial set $D_i$ is a Kan complex (or a
    $\infty$-category), the horn admits a
    filling $t\colon\Delta^n\rightarrow D_i$ for $0\leq k\leq n$ (respectively
    $0<k<n$), which gives us the commutative diagram
    \[\begin{tikzcd}
        {\Lambda^n_k} & {D_i} \\
        {\Delta^n} & {\colim_{\cI}D_i}
        \arrow["{\lambda_i}", from=1-2, to=2-2]
        \arrow["{\alpha_i}", from=1-1, to=1-2]
        \arrow["\alpha"{description, pos=0.25}, from=1-1, to=2-2]
        \arrow["t"', dotted, from=2-1, to=2-2]
        \arrow[from=1-1, to=2-1]
        \arrow["{t_i}"{description, pos=0.25}, dotted, from=2-1, to=1-2]
    \end{tikzcd}\]
    and in particular the $n$-simplex $t=\lambda_i\cdot t_i$ of
    $\colim_{\cI}D_i$ such that $t|_{\Lambda^n_k}=\lambda_i\cdot
    t_i|_{\Lambda^n_k}=\lambda_i\cdot\alpha_i=\alpha$.

    Now, if for every $i\in\cI$ the simplicial set $D_i$ is a Kan complex, this
    filling is always possible and therefore the colimit is itself a Kan
    complex. Similarly, if the simplicial sets in the diagram are
    $\infty$-categories the same goes for $\colim_{\cI}D_i$.
\end{proof}


\end{document}

\documentclass[a4paper,11pt,openany]{scrartcl}
\usepackage{reg2021}
\usepackage{quiver}

\begin{document}
\noindent\textbf{Authors}\hfill\textbf{Semester} \linebreak
\vspace*{-.1cm} Matteo Durante, Yuhao Zhang\hfill WS 2020/21 \\

\noindent
\rule{\linewidth}{1pt}
\begin{center}
\Large
\textbf{Higher Category Theory} \\
Assignment 5
\end{center}
\rule{\linewidth}{1pt}
\\

%%%%%%%%%%%%%%%%%%%%%%%%%%%%%%%%%%%%%%%%%%%%%%%%%%%%%%%%%%%%%%%%%%%%%%%%%%%%%%%

%%%sheet4
\newcommand{\La}{\Lambda}
\newcommand{\pa}{\partial}
\newcommand{\ob}{\operatorname{Ob}}
\newcommand{\mor}{\operatorname{Mor}}
\newcommand{\sto}{\twoheadrightarrow}

%%%sheet5
\newcommand{\plim}{\varprojlim}
\newcommand{\sst}{\subseteq}

\exercise{1}
\begin{proof}
    (1) Let $\cC=[3]$. We see that $N([3])=\Delta_3$, which has a non-degenerate
    3-simplex given by $\id_{\Delta_3}$. On the other hand, by definition all of
    the simplices of $Sk_2(\Delta_3)$ of dimension $>2$ are degenerate, hence
    the canonical inclusion $Sk_2(\Delta_3)\rightarrow \Delta_3$ is not an
    isomorphism.\\
\\
    (2) Since (co)limits of presheaves are defined pointwisely and a morphism of
    presheaves is a monomorphism if and only if every component of the natural
    transformation is, we only need to check that for all $a\in\Ob(\cA)$ the
    pushout squares
    \[\begin{tikzcd}
	{X_a} & {Y_a} \\
	{X'_a} & {Y'_a}
	\arrow["{i_a}"', from=1-1, to=2-1]
	\arrow["{i'_a}", from=1-2, to=2-2]
	\arrow["{g_a}"', from=2-1, to=2-2]
	\arrow["{f_a}", from=1-1, to=1-2]
    \end{tikzcd}\]
    are also pullback squares, so we shall be working solely in $\Set$, allowing
    us to drop the $a$, without creating ambiguity.

    Since the class of monomorphisms in $\Set$ is saturated, we know that $i'$
    is a monomorphism too. We will now verify that $X$ has the universal
    property of the pullback by exhibiting the universal property.

    Consider then $h_1\colon Z\rightarrow X'$, $h_2\colon Z\rightarrow Y$
    making the diagram commute. We are forced to define a candidate
    factorization $h\colon Z\rightarrow X$ by mapping $z\in Z$ to the unique
    $x\in X$ such that $h_1(z)=i(x)$,
    which grants us the uniqueness of an eventual factorization.
    By construction, $h$ is well-defined and $h_1=i\cdot h$, so we only have
    to check that $h_2=f\cdot h$. Notice that $i'\cdot h_2=g\cdot
    h_1=g\cdot i\cdot h=i'\cdot f\cdot h$ and, by injectivity of $i'$,
    we have the thesis.
\end{proof}

~\\
\exercise{2}
\begin{proof}
    (1) Once more, we only need to check that for all objects $a\in\Ob(\cA)$ the
    following is a coequalizer diagram.
    \[\begin{tikzcd}
	{X_a\times_{Y_a}X_a} & {X_a} & {\im(f)_a}
	\arrow["{p_a}", from=1-1, to=1-2, shift left=1]
	\arrow["{q_a}"', from=1-1, to=1-2, shift right=1]
	\arrow["{\pi_a}"', from=1-2, to=1-3, swap]
    \end{tikzcd}\]
    Here by $\pi$ we refer to the morphism we get from $f$ by restricting the
    codomain. $f\colon X\rightarrow Y$. From now on, like in the previous
    exercise, we shall work in $\Set$ and therefore drop every $a$.

    We begin by noticing that $\im(f)\cong X_{/\sim}$ under $\pi$, where $x\sim x'$
    whenever $f(x)=f(x')$, because $\pi$ is surjective by construction.

    Consider then a function $g\colon X\rightarrow Z$ coequalizing $p$ and $q$.
    All we have to do is show that, if $x\sim x'$, then $g(x)=g(x')$, since then
    $g$ will factor through $\pi\colon X\rightarrow X_{/\sim}$ as
    $\tilde{g}\colon X_{/\sim}\rightarrow Z$, $[x]\mapsto g(x)$. By construction,
    $\tilde{g}$ will coequalize $p$ and $q$, while the uniqueness of the
    factorization will follow from the surjectivity of $\pi$. To do this, we
    first characterize $X\times_YX$ explicitly.

    We claim that the pullback is given by $S=\{(x,x')\in X\times X\ |\
    f(x)=f(x')\}$ with the obvious projection maps $\pi_1(x,x')=x$,
    $\pi(x,x')=x'$. Indeed, consider a pair of maps $h_1,h_2\colon Z\rightarrow
    X$ such that $f\cdot h_1=f\cdot h_2$. Then, we may construct a factorization
    $h\colon Z\rightarrow S$ by setting $h(z):=(h_1(z),h_2(z))$. This is
    well-defined since $f(h_1(z))=(f\cdot h_1)(z)=(f\cdot h_2)(z)=f(h_2(z))$ and
    therefore $(h_1(z),h_2(z))\in S$. Also, by construction $\pi_i\cdot h=h_i$
    and the uniqueness of the factorization follows from the fact that these
    last equations (which are satisfied by all factorizations) specify both
    entries of a candidate $h(z)$.

    We now check that the $\tilde{g}$ we defined earlier is actually
    well-defined by checking that $x\sim x'$ implies $g(x)=g(x')$. This follows
    from the fact that $x\sim x'$ means $f(x)=f(x')$, thus $(x,x')\in
    X\times_YX$ and $g(x)=g(p(x,x'))=(g\cdot p)(x,x')=(g\cdot
    q)(x,x')=g(q(x,x'))=g(x')$.\\
\\
    (2) Suppose $T$ to be a representable presheaf, i.e.\ isomorphic to $\yo_a$
    for some $a\in\Ob(\cA)$. Since $\cA$ is small, $\hat{\cA}$ is locally small
    and therefore the hom-sets are actual sets. Writing equalities in place of
    natural isomorphisms, we have the following chain of identities:
    $\hat{\cA}(T,Y)=\hat{\cA}(\yo_a,Y)=Y_a=\bigcup_{i\in I}Y_{i,a}=\bigcup_{i\in
    I}\hat{\cA}(\yo_a,Y_i)=\bigcup_{i\in I}\hat{\cA}(T,Y_i)$. Here a natural
    transformation $s\colon T\cong\yo_a\rightarrow Y_i$ on the right is
    identified in $\bigcup_{i\in I}\hat{\cA}(T,Y_i)$ with all
    other natural transformations $s'\colon T\cong\yo_a\rightarrow Y_j$ such
    that $s=s'\in Y_a$ and the equality between the two extremes is exhibited by
    the map sending such a natural transformation $s\colon T\rightarrow Y_i$ to
    the one we get by composing with the inclusion $Y_i\rightarrow Y$, which is
    what we get if we follow the chain of identifications.
\end{proof}

~\\
\exercise{3}
\begin{proof}
(1) Recall that the nerve functor $N$ being a right adjoint, preserves products, and thus $\Delta^p\times\Delta^q\cong N([p]\times[q])$. For any $n$-simplex $s\colon\Delta^n\to\Delta^p\times\Delta^q$, under the adjunction
\[
\Hom_{\Cat}([n],[p]\times[q])\cong\Hom_{\sSet}(\Delta^n,\Delta^p\times\Delta^q)
\]
it corresponds to a unique $s'\colon[n]\to[p]\times[q]$. Suppose that $s$ is not a monomorphism. Then $s'$ is not either, which implies that $s'$ factorizes through some $[m]$ ($m<n$), say, into $[n]\overset{f'}{\to}[m]\overset{t'}{\to}[p]\times[q]$. Indeed, since the image $s'([n])\sst[p]\times[q]$ is a finite totally ordered set and $s'$ is not injective, there exists some $m<n$ such that $[m]\cong s'([n])$, and we may just take $f'$ to be the composition $[n]\to s'([n])\cong[m]$ and $t'\colon[m]\to[p]\times[q]$ to be the inclusion of a subset. Again $f',t'$ correspond to some $f\colon[n]\to[m]$ and $t\colon\Delta^m\to\Delta^p\times\Delta^q$ via the adjunction $\tau\dashv N$, and one has $s=tf=f^*(t)$. This shows that $s$ is degenerate. Hence the proof.\\
\\
(2) We claim that if $\Delta\to X$ and $\Delta^n\to Y$ are both degenerate, then so is $\Delta^n\to X\times Y$. To see this, assume they are degenerate and then $\Delta^n\to X$ and $\Delta^n\to Y$ factorize through $\Delta^k$, $\Delta^l$ for some $0\leqslant k,l<n$ respectively. Without loss of generality, one may further assume that $k\leqslant l$, then $\Delta^n\to\Delta^k$ factorizes through $\Delta^l$. We obtain a morphism $\Delta^k\to X\times Y$ by the universal property of products, through which $\Delta^n\to X\times Y$ factorizes, as depicted below:
\[
\begin{tikzcd}
\Delta^n\arrow[r]\arrow[d]& \Delta^l\arrow[dr]\arrow[from=dl]&\\
\Delta^k\arrow[dr]\arrow[r,dotted]& X\times Y\arrow[r]\arrow[d]\arrow[from=ul,crossing over]& Y\\
& X&
\end{tikzcd}
\]
Hence $\Delta^n\to X\times Y$ is degenerate, and this confirms our claim.

Therefore, if $\Delta^n\to X\times Y$ is non-degenerate, then either $\Delta^n\to X$ or $\Delta^n\to Y$ is degenerate, which implies that either $\Delta^n\to X$ or $\Delta^n\to Y$ is a monomorphism by the regularity of $X$ and $Y$. We thus may assume that $\Delta^n\to X$ is monic. Then by definition, $\Delta^n([m])\to X_m$ is an injective map of sets for all $m\geqslant 0$, and this in turn entails that
\[
\Delta^n([m])\to X_m\times Y_m=(X\times Y)_m
\]
is an injective map of sets. Consequently $\Delta^n\to X\times Y$ is a monomorphism.\\
\\
(3) Consider the diagram $F\colon I\to\sSet$ where $I$ is finite and $X^i:=F(i)$ is regular for each $i\in I$. Recall that finite limits can be exhibited by finite products and equalizers:
\[
\lim_IF=\ker\big(
\begin{tikzcd}[sep=small]
\prod_{i\in I}X_i\arrow[r,shift right]\arrow[r,shift left]&\prod_{i\to j}X_i
\end{tikzcd}\big)
\]
and by (2) plus induction we know that $\prod_{i\in I}X_i$ is regular if each $X_i$ is. 

Thus the case is reduced to equalizers: in other words, it suffices to show that for any diagram $X\rightrightarrows Y$ in $\sSet$, the equalizer
\[
K:=\ker\big(X\rightrightarrows Y\big)
\]
is a regular simplicial set if $X$ is so. To this end, suppose that an $n$-simplex $s\colon\Delta^n\to K$ is not a monomorphism. Then the composition $\Delta^n\to K\to X$ is not a monomorphism (since it will not be injective over some $[l]$) as well, and by the fact that $X$ is regular, the composite $\Delta^n\to X$ factors through some $\Delta^m$.
\[
\begin{tikzcd}
K\arrow[r]& X\arrow[r,shift right]\arrow[r,shift left]& Y\\
\Delta^n\arrow[u,"s"']\arrow[ur,dotted]\arrow[d,"t"]&&\\
\Delta^m\arrow[uur,dotted]\arrow[uu,bend left=45,dotted,"u"]&&
\end{tikzcd}
\]
From this we can see that $\Delta^m\to X$ equalizes $X\rightrightarrows Y$, and by the universal property of equalizers, there is a unique morphism $u\colon\Delta^m\to K$. Using the universal property of equalizers again yields that $ut=s$, which means $s\colon\Delta^n\to K$ being degenerate.
\end{proof}

\end{document}

\documentclass[a4paper,11pt,openany]{scrartcl}
\usepackage{../reg2021}
\usepackage{../quiver}

\begin{document}
\noindent\textbf{Author}\hfill\textbf{Semester} \linebreak
\vspace*{-.1cm} Matteo Durante, Yuhao Zhang\hfill WS 2020/21 \\

\noindent
\rule{\linewidth}{1pt}
\begin{center}
\Large
\textbf{Higher Category Theory} \\
Assignment 9
\end{center}
\rule{\linewidth}{1pt}
\\

%%%%%%%%%%%%%%%%%%%%%%%%%%%%%%%%%%%%%%%%%%%%%%%%%%%%%%%%%%%%%%%%%%%%%%%%%%%%%%%

%%%sheet4
\newcommand{\La}{\Lambda}
\newcommand{\pa}{\partial}
\newcommand{\ob}{\operatorname{Ob}}
\newcommand{\mor}{\operatorname{Mor}}
\newcommand{\sto}{\twoheadrightarrow}

%%%sheet5
\newcommand{\plim}{\varprojlim}
\newcommand{\sst}{\subseteq}
\newcommand{\eq}{\operatorname{eq}}

%%%sheet6
\newcommand{\f}{\varphi}

%%%sheet8
\newcommand{\sing}{\operatorname{Sing}}

%%%sheet9
\newcommand{\ihom}{\underline{\Hom}}


~\\
\exercise{1}
\begin{proof}
(1) It is enough show that any two $i,j\colon\Delta^0\to\Delta^n$ are $\Delta^1$-homotopic. For this, define a map $\Delta^1\cong\Delta^0\times\Delta^1\to\Delta^n$ by $h\colon[1]\to[n]$, where $h(0):=i(0)$ and $h(1):=j(1)$. Then it is a $\Delta^1$-homotopy connecting $i$ and $j$.

(2) Let us consider $i,j\in\{0,\cdots,n\}$. Then by (1) we know that there is a $\Delta^1$-homotopy connecting $i$ and $j$. Then the composite $\Delta^1\overset{h}{\to}\Delta^n\overset{s}{\to}X$ gives a $\Delta^1$-homotopy connecting $si$ and $sj$. Hence $[si]=[sj]$.

(3) Let us denote by $C$ the functor $\Set\to\sSet$ sending a set $E$ to the constant presheaf with value $E$. To show the adjunction $\pi_0\dashv C$, it suffices to check that for each simplicial set $X$, the functor $\Hom_{\sSet}(X,C(-))$ is represented by $\pi_0(X)$. We define a map
\[
\Phi\colon\Hom_{\Set}(\pi_0(X),E)\to\Hom_{\sSet}(X,C(E))
\]
by sending each $f\colon\pi_0(X)\to E$ to the simplicial map $\Phi(f)$ given by $\Phi(f)_n\colon X_n\to C(E)_n=E$, $(s\colon\Delta^n\to X)\mapsto f([si])$, where $i\in\{0,\cdots,n\}$ is arbitrary. Its well-definedness comes from (ii). We assert that $\Phi$ has an inverse
\[
\Big(g_*\colon\pi_0(X)\to\pi_0(C(E))\cong E\Big)\mapsfrom\Big(g\colon X\to C(E)\Big):\Psi
\]
To check $\Psi$ well-defined, it suffices to see that $\pi_0(C(E))\cong E$, while this is obvious, since $C(E)$ is constant with value $E$ and so $\pi_0(C(E))\cong\colim_{\Delta^{\op}} C(E)\cong E$. Verifying $\Phi$ and $\Psi$ being mutually inverse is straightforward. For example, for any $f\colon\pi_0(X)\to E$ and $s\in X_0$, we have
\[
(\Psi\Phi(f))([s])=\Phi(f)_*([s])=(\Phi(f)\circ s)_0(0)=\Phi(f)_0(s_0(0))=\Phi(f)_0(s)=f([s(0)])=f([s]),
\]
where the first equality is seen by noting that $[\Delta^0,C(E)]=\pi_0(C(E))\cong E$ is explicitly given by $[s]\mapsto s_0(0)$. It remains to show that the bijection $\Phi$ is functorial in $E$, while this is obvious via the definition of $\Psi$.

(4) Let us first recall that by Yoneda, we have
\[
\Hom_{\sSet}(\Delta^0,\ihom(X,Y))\cong\ihom(X,Y)_0=\Hom_{\sSet}(X,Y)
\]
which sends any $f\colon\Delta^0\to\ihom(X,Y)$ to $f_0(0)$. Hence to prove $\pi_0(\ihom(X,Y))=[X,Y]$, it is enough to show that $f\sim g$ if and only if $f_0(0)\sim g_0(0)$ for any simplicial maps $f,g\colon\Delta^0\to\ihom(X,Y)$. Since the equivalence relation ``$\sim$'' is generated by the (reflexive and symmetric) relation ``connected by a $\Delta^1$-homotopy'', $f\sim g$ if and only if there are $f_1,\cdots,f_n$ for some integer $n$ and $\Delta^1$-homotopies from $f$ to $f_1$,..., from $f_{n-1}$ to $f_n$, and from $f_n$ to $g$. Thus the case is reduced to prove that $f$ and $g$ are connected by a $\Delta^1$-homotopy if and only if $f_0(0)$ and $g_0(0)$ are so. However, this can be seen by using Yoneda again, as follows:
\[
\begin{tikzcd}
\Hom_{\sSet}(\Delta^1,\ihom(X,Y))\arrow[r,equal,"\sim"]\arrow[d,"0_*",shift left]\arrow[d,"1_*"',shift right]& \Hom_{\sSet}(\Delta^1\times X,Y)\arrow[d,dotted,shift left]\arrow[d,dotted,shift right]\\
\Hom_{\sSet}(\Delta^0,\ihom(X,Y))\arrow[r,equal,"\sim"]& \Hom_{\sSet}(X,Y)
\end{tikzcd}
\] 
If there is a $\Delta^1$-homotopy $h\colon\Delta^1\to\ihom(X,Y)$ with $h_0=f$ and $h_1=g$, then by Yoneda we get a simplicial map $h'\colon\Delta^1\times X\to Y$, and from the diagram above one sees that $h'_0=f_0(0)$ and $h'_1=g_0(0)$, and vice versa.

(5) Denote by $\cF$ the class of maps inducing a bijection after applying $\pi_0$. First of all, we observe that $\cF$ is stable under retracts. Indeed, if $f\colon K\to L$ is in $\cF$ and admits a retract $g\colon X\to Y$, then applying $\pi_0$ yields a commutative diagram
\[
\begin{tikzcd}
\pi_0(X)\arrow[r,"s"]\arrow[d,"g_*"]& \pi_0(K)\arrow[d,"f_*", "\wr"']\arrow[r,"p"]& \pi_0(X)\arrow[d,"g_*"]\\
\pi_0(Y)\arrow[r,"t"']& \pi_0(L)\arrow[r,"q"']& \pi_0(Y)
\end{tikzcd}
\]
where $ps=\id$, $qt=\id$ and $f_*$ is a bijection. From $pf_*^{-1}tg_*=ps=\id$, one gets that $g_*$ is injective, while from $g_*pf_*^{-1}t=qt=\id$, it follows that $g_*$ is surjective. Hence $g_*$ is a bijection, i.e. $g\in\cF$. 

Moreover, we claim that $\cF$ is closed under colimits, and hence under pushouts, coproducts and countable compositions. For this, take any $f_i\colon K_i\to L_i$ indexed by some small category $I$ with $f_i\in\cF$. By Exercise 1(i) of Sheet 7, we have $[\Delta^0,X]=\colim_{\Delta^{\op}}X$ for any simplicial set $X$ (because any $s,t\in X_0$ being connected by a $\Delta^1$-homotopy is the same as saying that there is a path in $X_1$ connecting $s$ and $t$). Then we get a bijection 
\[
\colim_If_{i*}=\colim_I\colim_{\Delta^{\op}}f_i=\colim_{\Delta^{\op}}\colim_If_i=(\colim_If_i)_*
\]
so that $\colim_If_i\in\cF$. Therefore the class $\cF$ is saturated.

It remains to show that $\{i\}\times K\subset\Delta^1\times K$ lies in $\cF$ for any simplicial set $K$. That is, to prove that the induced map
\[
[\Delta^0,\{i\}\times K]\to[\Delta^0,\Delta^1\times K]
\]
is a bijection. For this, it is enough to show that any two maps $\Delta^0\to\Delta^1\times K$ represented by $(0,k)$ and $(1,k)$ ($k\in K_0$) respectively are $\Delta^1$-homotopic. However, this is obvious, since $(\id_{[1]},s^0_1(k))\colon\Delta^1\to\Delta^1\times K$ gives a $\Delta^1$-homotopy from $(0,k)$ to $(1,k)$.

Now use Gabriel-Zisman, and we know that anodyne extensions are in $\cF$.
\end{proof}

\exercise{2}
\begin{proof}
    $(1)$ Remembering that the map $I\times A\cup\{0\}\times B\rightarrow
    I\times B$ induced by the monomorphism $i$ is a $(I,S)$-anodyne extension,
    we construct the square
    \[\begin{tikzcd}
        {I\times A\cup\{0\}\times B} && X \\
        {I\times B} & B & Y
        \arrow["{h\cup f}", from=1-1, to=1-3]
        \arrow["{j}", hook, from=1-1, to=2-1]
        \arrow["{pr_2}"', from=2-1, to=2-2]
        \arrow["b"', from=2-2, to=2-3]
        \arrow["p", from=1-3, to=2-3]
        \arrow["s"{description}, dotted, from=2-1, to=1-3]
    \end{tikzcd},\]
    which is possible since $h|_{\{0\}\times A}=h_0=f\cdot i=f|_A$. It commutes
    because
    \begin{align*}
        p\cdot(h\cup f) &=(p\cdot h)\cup (p\cdot f) \\
        &=(p\cdot a\cdot pr_2)\cup b \\
        &=(b\cdot i\cdot pr_2)\cup b \\
        &=(b\cdot pr_2\cdot(\id_I\times i))\cup b \\
        &=b\cdot((pr_2\cdot(\id_I\times i))\cup\id_B) \\
        &=b\cdot pr_2\cdot j,
    \end{align*}
    hence there is a filling $s\colon I\times B\rightarrow X$ as pictured.
    We now choose $g=s|_{\{1\}\times B}$. By construction,
    \begin{align*}
        p\cdot g &=p\cdot s|_{\{1\}\times B} \\
        &=b\cdot pr_2|_{\{1\}\times B} \\
        &=b
    \end{align*}
    and
    \begin{align*}
        g\cdot i &=s|_{\{1\}\times B}\cdot i \\
        &=s\cdot(\id_I\times i)|_{\{1\}\times A} \\
        &=h|_{\{1\}\times A} \\
        &=h_1 \\
        &=a,
    \end{align*}
    which proves that the $g$ we constructed has the desired properties.

    $(2)$ We first construct a constant homotopy $h'$ from $a$ to $a$ by
    setting $h':=a\cdot pr_2\colon I\times A\rightarrow X$. Seeing $\partial
    I\times A$, $\partial
    I\times B$ as $A\sqcup A$, $B\sqcup B$, we can construct the diagram
    \[\begin{tikzcd}
        {I\times A\cup\partial I\times B} && X \\
        {I\times B} & B & Y
        \arrow["p", from=1-3, to=2-3]
        \arrow["{h'\cup(f_0\sqcup f_1)}", from=1-1, to=1-3]
        \arrow["{j}", from=1-1, to=2-1]
        \arrow["{pr_2}"', from=2-1, to=2-2]
        \arrow["b"', from=2-2, to=2-3]
        \arrow["h"{description}, dashed, from=2-1, to=1-3]
    \end{tikzcd},\]
    which is possible because $h'|_{\partial I\times A}=a\sqcup a=f_0\sqcup
    f_1|_{\partial I\times A}$ by definition. It also commutes because
    \begin{align*}
        p\cdot(h'\cup(f_0\sqcup f_1))&=(p\cdot h')\cup ((p\cdot f_0)\cup(p\cdot
        f_1)) \\
        &=(p\cdot a\cdot pr_2)\cup(b\sqcup b) \\
        &=(b\cdot i\cdot pr_2)\cup(b\sqcup b) \\
        &=b\cdot ((i\cdot pr_2)\cup(\id_B\sqcup\id_B)) \\
        &=b\cdot((pr_2\cdot (\id_I\times i))\cup(\id_B\sqcup\id_B)) \\
        &=b\cdot pr_2\cdot j
    \end{align*}
    Recall now that, since $i$ is a $(I,S)$-anodyne map, so is $j$, hence our
    square admits the depicted filling $h\colon I\times B\rightarrow X$, which
    will be our desired homotopy from $f_0$ to $f_1$. Indeed, $h|_{\partial
    I\times B}=f_0\sqcup f_1$ and $h|_{I\times A}=h'$, that is it is constant on
    $A$. We still have to show that it is also constant over $Y$, but this
    follows again by construction from $p\cdot h=b\cdot pr_2$, hence the thesis.
\end{proof}

~\\
\exercise{3}
\begin{proof}
    First of all remember that, fixed a monomorphism $i\colon K\rightarrow L$ in
    $\Set\cong\widehat{[1]}$, for $\epsilon =0,1$ the induced map $I\times
    K\cup\{\epsilon\}\times L\rightarrow I\times L$ is $(I,S)$-anodyne.
    This map comes from the pushout square
    \[\begin{tikzcd}
        {\{\epsilon\}\times K} & {\{\epsilon\}\times L} \\
        {I\times K} & {I\times K\cup\{\epsilon\}\times L} \\
        && {I\times L}
        \arrow[curve={height=-12pt}, from=1-2, to=3-3]
        \arrow[curve={height=12pt}, from=2-1, to=3-3]
        \arrow["j", dashed, from=2-2, to=3-3]
        \arrow[from=1-1, to=1-2]
        \arrow[from=1-1, to=2-1]
        \arrow[from=2-1, to=2-2]
        \arrow[from=1-2, to=2-2]
    \end{tikzcd}\]
    inducing the pictured factorization.

    Since $I\cong 2$, studying the pushout we
    get $I\times K\cup\{\epsilon\}\times L=(K\sqcup K) \cup
    (\emptyset\sqcup L)=K\sqcup L$ for $\epsilon=1$ from a previous exercise and
    $I\times L=L\sqcup L$. Also, the map $j\colon K\sqcup L\rightarrow L\sqcup
    L$ is simply the inclusion $i\sqcup\id_L$. Assuming that $\emptyset\neq
    K\subset L$, we will now show that $i$ is a retract of this map. In order to
    do this, fix $k\in K$ and construct the diagram
    \[\begin{tikzcd}
        K & {K\sqcup L} & K \\
        L & {L\sqcup L} & L
        \arrow["{in_0}", from=1-1, to=1-2]
        \arrow["{\id_K+k}", from=1-2, to=1-3]
        \arrow["i"', from=1-3, to=2-3]
        \arrow["{i\sqcup\id_L}"', from=1-2, to=2-2]
        \arrow["{\id_L+k}"', from=2-2, to=2-3]
        \arrow["{in_0}"', from=2-1, to=2-2]
        \arrow["i"', from=1-1, to=2-1]
    \end{tikzcd},\]
    which proves our claim.

    Since $(I,S)$-anodyne maps form a saturated class, it follows that $i$ is
    one as well when $K$ (and therefore $L$) is not the empty set. Notice that
    we didn't mention the small set $S$ at all.
\end{proof}

\end{document}

\documentclass[a4paper,11pt,openany]{scrartcl}
\usepackage{../reg2021}
\usepackage{../quiver}

\begin{document}
\noindent\textbf{Author}\hfill\textbf{Semester} \linebreak
\vspace*{-.1cm} Matteo Durante, Yuhao Zhang\hfill WS 2020/21 \\

\noindent
\rule{\linewidth}{1pt}
\begin{center}
\Large
\textbf{Higher Category Theory} \\
Assignment 10
\end{center}
\rule{\linewidth}{1pt}
\\

%%%%%%%%%%%%%%%%%%%%%%%%%%%%%%%%%%%%%%%%%%%%%%%%%%%%%%%%%%%%%%%%%%%%%%%%%%%%%%%

%%%sheet4
\newcommand{\La}{\Lambda}
\newcommand{\pa}{\partial}
\newcommand{\ob}{\operatorname{Ob}}
\newcommand{\mor}{\operatorname{Mor}}
\newcommand{\sto}{\twoheadrightarrow}

%%%sheet5
\newcommand{\plim}{\varprojlim}
\newcommand{\sst}{\subseteq}
\newcommand{\eq}{\operatorname{eq}}

%%%sheet6
\newcommand{\f}{\varphi}

%%%sheet8
\newcommand{\sing}{\operatorname{Sing}}

%%%sheet9
\newcommand{\ihom}{\underline{\Hom}}


~\\
\exercise{1}
\begin{proof}
(1) Denote by $\cF$ the class of maps being sent to bijections through $\pi_0$. Firstly we observe that $\cF$ is stable under retracts. Indeed, if $f\colon K\to L$ is in $\cF$ and admits a retract $g\colon X\to Y$, then applying $\pi_0$ yields a commutative diagram
\[
\begin{tikzcd}
\pi_0(X)\arrow[r,"s"]\arrow[d,"g_*"]& \pi_0(K)\arrow[d,"f_*", "\wr"']\arrow[r,"p"]& \pi_0(X)\arrow[d,"g_*"]\\
\pi_0(Y)\arrow[r,"t"']& \pi_0(L)\arrow[r,"q"']& \pi_0(Y)
\end{tikzcd}
\]
where $ps=\id$, $qt=\id$ and $f_*$ is a bijection. From $pf_*^{-1}tg_*=ps=\id$, one gets that $g_*$ is injective, while from $g_*pf_*^{-1}t=qt=\id$, it follows that $g_*$ is surjective. Hence $g_*$ is a bijection, i.e. $g\in\cF$. 

Moreover, we claim that $\cF$ is closed under colimits, and hence under pushouts, coproducts and countable compositions. To this end, take any $f_i\colon K_i\to L_i$ in $\cF$ indexed by a small category $I$. Since $\pi_0$ is a left adjoint, we have $\pi_0(\colim_If_i)=\colim_I\pi_0(f_i)$ is a bijection and thus $\colim_If_i\in\cF$. Therefore $\cF$ is saturated.

(2) Recall that
\[
(\La^n_k)_i=\{f\colon[i]\to[n]\mid\im(f)\not\supseteq\{0,\cdots,k-1,k+1,\cdots,n\}\}
\]
for any $i$. Hence it follows directly that $(\La^n_k)_i=\Delta^n_i$ for $n\geqslant2$ and $i=0,1$. Therefore $\pi_0(\La^n_k)=[\Delta^0,\La^n_k]\cong[\Delta^0,\Delta^n]=\pi_0(\Delta^n)$ for $n\geqslant2$. For $n=1$ we have $\La^1_0=\La^1_1=\Delta^0$ and $\pi_0(\La^1_k)=*$, while by Exercise 1.1 of Sheet 9 we know that $\pi_0(\Delta^n)=*$ for any $n$. Nevertheless, notice that this is not true for $n=0$, as the $0$-horn $\La^0_0=\emptyset$ but $\pi_0(\Delta^0)=*$.

(3) From (2) it follows that the inclusions $\La^n_k\hookrightarrow\Delta^n$ (for $n\geqslant1$ and $0\leqslant k\leqslant n$) are in $\cF$. Hence by Gabriel-Zisman all anodyne extensions belong to $\cF$.

(4) This follows immediately from (3) and a theorem in Lecture 17.

(5) Let us suppose first that $X$ and $Y$ are Kan complexes. We define a map 
\[
\pi_0(X\times Y)\to\pi_0(X)\times\pi_0(Y)
\]
by sending $[(x_0,y_0)]\mapsto([x_0],[y_0])$ for all $x_0\in X_0$, $y_0\in Y_0$. It is well-defined, since if $(x_0,y_0)\sim(x_1,y_1)$, then due to $X\times Y$ being a Kan complex there is a $\Delta^1$-homotopy $h\colon\Delta^1\to X\times Y$ with $h_0=(x_0,y_0)$ and $h_1=(x_1,y_1)$, so that $x_0\sim x_1$ via $p_Xh$ and $y_0\sim y_1$ via $p_Yh$ (where $p_X$, $p_Y$ are the projections from $X\times Y$ to $X$, $Y$). The surjectivity is evident, because any $([x_0],[y_0])$ admits a preimage $[(x_0,y_0)]$. We note that it is also injective. In fact, if $([x_0],[y_0])=([x_1],[y_1])$, then there are $\Delta^1$-homotopies $h_X\colon\Delta^1\to X$ connecting $x_0$, $x_1$ and $h_Y\colon\Delta^1\to Y$ connecting $y_0$, $y_1$. The universal property of products gives a simplicial map $h\colon\Delta^1\to X\times Y$. Since $p_{X0}h_0=(p_Xh)_0=(h_X)_0=x_0$ and $p_{Y0}h_0=(p_Yh)_0=(h_Y)_0=y_0$, we have $h_0=(x_0,y_0)$. Similarly $h_1=(x_1,y_1)$, which shows that $[(x_0,y_0)]=[(x_1,y_1)]$.

For the general case, recall that $(\text{anodyne extension},\text{Kan fibration})$ is a weak factorization system, so we can find anodyne extensions $X\to X'$ and $Y\to Y'$ where $X',Y'$ are Kan complexes. Then $X\times Y\to X'\times Y'$ is a weak homotopy equivalence (Lecture 18), and by (4) we conclude that
\[
\pi_0(X\times Y)\cong\pi_0(X'\times Y')\cong\pi_0(X')\times\pi_0(Y')\cong\pi_0(X)\times\pi_0(Y).
\]
This finishes the proof.
\end{proof}

~\\
\exercise{2}
\begin{proof}
    (1) We begin by considering a commutative diagram
    \[\begin{tikzcd}
        {\Lambda^n_k} & {p^{-1}(a)=X_a} & X \\
        {\Delta^n} & {\Delta^0} & A
        \arrow["p", from=1-3, to=2-3]
        \arrow["a"', from=2-2, to=2-3]
        \arrow[from=1-2, to=1-3]
        \arrow[from=1-2, to=2-2]
        \arrow[from=2-1, to=2-2]
        \arrow[from=1-1, to=2-1]
        \arrow[from=1-1, to=1-2]
        \arrow[dotted, from=2-1, to=1-3]
        \arrow[dotted, from=2-1, to=1-2]
        \arrow["\lrcorner"{anchor=center, pos=0.01}, draw=none, from=1-2, to=2-3]
    \end{tikzcd},\]
    where $0\leq k<n$ and the square on the right is a pullback. From the LLP of
    $\Lambda^n_k\rightarrow\Delta^n$ against $p$ we get a lift
    $\Delta^n\rightarrow X$ and then, using the universal property of the
    pullback with respect to the lift and $\Delta^n\rightarrow\Delta^0$, we get
    a lift of $\Lambda^n_k\rightarrow\Delta^n$ against $X_a\rightarrow\Delta^0$.

    This implies that $X_a$ is an $\infty$-category, hence we only need to prove
    that its morphisms are invertible, which will make it a $\infty$-groupoid
    and therefore a Kan complex.

    To prove this, for any morphism $f\colon x\rightarrow y$ in $X_a$ we
    consider the diagram
    \[\begin{tikzcd}
        {\Lambda^2_0} & {X_a} \\
        {\Delta^2}
        \arrow["{(\id_x,f)}", from=1-1, to=1-2]
        \arrow[from=1-1, to=2-1]
        \arrow["t"', dotted, from=2-1, to=1-2]
    \end{tikzcd}\]
    inducing the pictured 2-simplex $t$ by our previous observations. The
    morphism $f$ is a right inverse of $d^2(t)=g\colon y\rightarrow x$ and from
    \[\begin{tikzcd}
        {\Lambda^2_0} & {X_a} \\
        {\Delta^2}
        \arrow["{(\id_y,g)}", from=1-1, to=1-2]
        \arrow[from=1-1, to=2-1]
        \arrow["u"', dotted, from=2-1, to=1-2]
    \end{tikzcd}\]
    we also get a left inverse $d^2(u)=h$ of $g$. It follows that $g$ is
    invertible and the same goes for $f$.

    (2) Let's consider for any morphism $f\colon a_0\rightarrow a_1$ in $A$ the
    commutative diagram
    \[\begin{tikzcd}
        {\Lambda^1_0=\Delta^0} & X \\
        {\Delta^1} & A
        \arrow["p", from=1-2, to=2-2]
        \arrow["f"', from=2-1, to=2-2]
        \arrow["{x_0}", from=1-1, to=1-2]
        \arrow[from=1-1, to=2-1]
        \arrow["{\phi}"{description}, dotted, from=2-1, to=1-2]
    \end{tikzcd},\]
    which from the LLP of $\Lambda^1_0\rightarrow\Delta^1$ against $p$ grants us
    the desired lift $\phi\colon x_0\rightarrow x_1$ of $f$ along $p$.

    To prove that the equivalence class of $x_1$ in $\pi_0(X_{a_1})$ does not
    depend on the choice of the lift we consider for any other such lift
    $\psi\colon x_0\rightarrow y$ the commutative diagram
    \[\begin{tikzcd}
        {\Lambda^2_0} & X \\
        {\Delta^2} & A
        \arrow["{s_0(f)}"', from=2-1, to=2-2]
        \arrow["{(\psi,\phi)}", from=1-1, to=1-2]
        \arrow["p", from=1-2, to=2-2]
        \arrow[from=1-1, to=2-1]
        \arrow["t"{description}, dotted, from=2-1, to=1-2]
    \end{tikzcd},\]
    granting us a 2-simplex $t$ which induces a morphism $d^0(t)=\xi\colon
    x_1\rightarrow y$. The commutative diagram
    \[\begin{tikzcd}
        {\Delta^1} \\
        & {X_a} & X \\
        & {\Delta^0} & A
        \arrow[curve={height=6pt}, from=1-1, to=3-2]
        \arrow["\xi", curve={height=-6pt}, from=1-1, to=2-3]
        \arrow["\xi"{description}, dotted, from=1-1, to=2-2]
        \arrow[from=2-2, to=3-2]
        \arrow["{a_1}", from=3-2, to=3-3]
        \arrow["p", from=2-3, to=3-3]
        \arrow[from=2-2, to=2-3]
        \arrow["\lrcorner"{anchor=center, pos=0.01}, draw=none, from=2-2, to=3-3]
    \end{tikzcd}\]
    then shows that this morphism also lies in $X_a$ through the universal
    property of the pullback and therefore $[x_1]=[y]$ in $\pi_0(X_a)$.

    We have just proven that distinct lifts through $p$ of the same morphism
    have the same codomain up to equivalence as long as they have the same
    domain.

    (3) Let $t\colon\Delta^2\rightarrow A$ be the map corresponding to our
    commutative trangle. We proceed by drawing the commutative diagram
    \[\begin{tikzcd}
        {\Lambda^2_1} & X \\
        {\Delta^2} & A
        \arrow["t"', from=2-1, to=2-2]
        \arrow["p", from=1-2, to=2-2]
        \arrow["{(\phi',\phi)}", from=1-1, to=1-2]
        \arrow[from=1-1, to=2-1]
        \arrow["u"{description}, dotted, from=2-1, to=1-2]
    \end{tikzcd},\]
    which by the LLP of $\Lambda^2_0\rightarrow\Delta^2$ against $p$ grants us a
    lift $u\colon\Delta^2\rightarrow X$ (and therefore a commutative triangle)
    with $d^0(u)=\phi'$, $d^1(u)=\psi\colon x_0\rightarrow x_2$ and
    $d^2(u)=\phi$ such that $p(\psi)=g$.

    (4) The functor, which we will denote by $F$, has already been well defined
    on objects, hence we only need to specify the action on morphisms. We know
    that for any map $f\colon a_0\rightarrow a_1$ in $A$ we have a lift
    $\phi\colon x_0\rightarrow x_1$ such that $p(\phi)=f$, thus we define
    $F([f])\colon\pi_0(X_{a_0})\rightarrow\pi_0(X_{a_1})$ as
    $F([f])([x_0])=[x_1]$, where $[x_1]$ lies in $\pi_0(X_{a_1})$ since
    $p(d^0(\phi))=d^0(p(\phi))=d^0(f)=a_1$.
    We need to show that this map is well defined, for which we will start with
    proving that, after fixing a representative $f$ of $[f]$, if we have a
    morphism $\psi\colon x_0\rightarrow x_0'$ in $X_{a_0}$ then we also have a
    morphism $x_1\rightarrow x_1'$ in $X_{a_1}$ between the objects specified by
    the liftings $\phi$, $\phi'$ of $f$ with domains $x_0$, $x_0'$.

    We can construct a map
    $(\phi'\cdot\psi,\phi)\colon\Lambda^2_0\rightarrow X$ which, composed
    with $p$, gives us $(p(\phi'\cdot\psi),f)\colon\Lambda^2_0\rightarrow A$. We
    want to extend this to a $2$-simplex $t\colon\Delta^2\rightarrow A$ where
    $d^0(t)=\id_a$; we will then lift it through $p$ thanks to the RLP with
    respect to $\Lambda^2_0\rightarrow\Delta^2$, getting a 2-simplex $u$ in $X$
    such that $d^0(u)$ is by construction the desired morphism $x_1\rightarrow
    x_1'$ in $X_{a_1}$.
    \[\begin{tikzcd}
        {\Lambda^2_0} & X \\
        {\Delta^2} & A
        \arrow["{(\phi'\cdot\psi,\phi)}", from=1-1, to=1-2]
        \arrow["p", from=1-2, to=2-2]
        \arrow["t"', from=2-1, to=2-2]
        \arrow[from=1-1, to=2-1]
        \arrow["{u}"{description}, dotted, from=2-1, to=1-2]
    \end{tikzcd}\]

    Notice that we have 2-simplices $v$, $v'$ showing that
    $f\cdot\id_a=p(\phi')\cdot p(\psi)\sim p(\phi'\cdot\psi)$, $f\cdot\id_a\sim
    f$, thus we may construct a horn $(s_0(f),v',v)\colon\Lambda^3_1\rightarrow
    A$ which can be extended to a 3-simplex $\alpha$ such that $d^1(\alpha)=t$
    is the desired 2-simplex in $A$.

    Having proven that $F([f])([x_0])$ does not depend on the representative of
    $[x_0]$, we show that it also does not depend on the representative of
    $[f]$.

    Suppose that $g\in [f]$, i.e.\ we have a 2-simplex $t$ in $A$ showing that
    $\id_a\cdot f\sim g$, meaning that $d^0(t)=\id_a$, $d^1(t)=g$, $d^2(t)=f$.
    After choosing lifts $\phi\colon x_0\rightarrow x_1$, $\psi\colon
    x_0\rightarrow x_1'$ of $f$, $g$ through $p$, we can construct the
    commutative square
    \[\begin{tikzcd}
        {\Lambda^2_0} & X \\
        {\Delta^2} & A
        \arrow["{(\psi,\phi)}", from=1-1, to=1-2]
        \arrow[from=1-1, to=2-1]
        \arrow["t"', from=2-1, to=2-2]
        \arrow["p", from=1-2, to=2-2]
        \arrow["u"{description}, dotted, from=2-1, to=1-2]
    \end{tikzcd},\]
    where the lift $u$ is such that $d^0(u)=h$ provides the desired morphism
    $x_1\rightarrow x_1'$ in $X_{a_1}$.

    This shows that $F([f])$ is well defined. We still have to prove that this
    association is functorial.

    If $[f]=[\id_a]$, then for any $[x]\in\pi_0(X_a)$ we may pick $\id_x$ as a
    lift of $\id_a$ through $p$, which then shows that $F([\id_a])([x])=[x]$.

    On the other hand, consider two composable morphisms $[f]$, $[g]$, where
    $\dom(f)=a$. Given a
    2-simplex $t$ in $A$ such that $d^0(t)=g$, $d^1(t)=g\cdot f$, $d^2(t)=f$ and
    fixed an element $[x_0]\in\pi_0(X_a)$, after fixing lifts $\phi\colon
    x_0\rightarrow x_1$, $\psi\colon x_1\rightarrow x_2$ of $f$, $g$ by (3) we
    get a 2-simplex $u$ in $X$
    such that $d^0(u)=\psi$, $d^1(u)=\xi\colon x_0\rightarrow x_2$,
    $d^2(u)=\phi$ and $\xi$ is a lift of $g\cdot f$ through $p$ with
    $\phi\cdot\psi\sim\xi$. It follows that $F([g]\cdot [f])=F([g])\cdot
    F([f])$.
\end{proof}

\end{document}

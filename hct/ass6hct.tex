\documentclass[a4paper,11pt,openany]{scrartcl}
\usepackage{../reg2021}
\usepackage{../quiver}

\begin{document}
\noindent\textbf{Authors}\hfill\textbf{Semester} \linebreak
\vspace*{-.1cm} Matteo Durante, Yuhao Zhang\hfill WS 2020/21 \\

\noindent
\rule{\linewidth}{1pt}
\begin{center}
\Large
\textbf{Higher Category Theory} \\
Assignment 5
\end{center}
\rule{\linewidth}{1pt}
\\

%%%%%%%%%%%%%%%%%%%%%%%%%%%%%%%%%%%%%%%%%%%%%%%%%%%%%%%%%%%%%%%%%%%%%%%%%%%%%%%

%%%sheet4
\newcommand{\La}{\Lambda}
\newcommand{\pa}{\partial}
\newcommand{\ob}{\operatorname{Ob}}
\newcommand{\mor}{\operatorname{Mor}}
\newcommand{\sto}{\twoheadrightarrow}

%%%sheet5
\newcommand{\plim}{\varprojlim}
\newcommand{\sst}{\subseteq}
\newcommand{\eq}{\operatorname{eq}}

%%%sheet6
\newcommand{\f}{\varphi}

\exercise{1}
\begin{proof}
(1) From definition one sees that $\Ob(\cA)=\Ob(\cA_0)\cup\Ob(\cA_1)$. We construct the functor $u\colon\cA\to\cC*\cD$ as follows: on objects, 
\[
u(a):=\left\{
\begin{array}{ll}
u_0(a),&\text{if }a\in\Ob(\cA_0)\\
u_1(a),&\text{if }a\in\Ob(\cA_1)
\end{array}
\right.
\]
and on morphisms,
\[
u(a\to b):=\left\{
\begin{array}{ll}
u_i(a\to b),&\text{if } a,b\in\Ob(\cA_i),\ i=0,1\\
0,&\text{if $a\in\Ob(\cA_0)$ and $b\in\Ob(\cA_1)$}.
\end{array}
\right.
\]
Note that there is no $a\to b$ with $a\in\Ob(\cA_1)$ and $b\in\Ob(\cA_0)$, since otherwise applying $q\colon\cA\to[1]$ to it yields a morphism $1\to0$. From the definition it follows that the restriction of $u$ to $\cA_0$ and $\cA_1$ are $u_0$ and $u_1$ respectively. Next we check that $pu=q$. Indeed, we have $pu(a)=pu_i(a)=i=q(a)$ for $a\in\Ob(\cA_i)$ ($i=0,1$), and
\[
pu(a\to b)=\left\{
\begin{array}{ll}
pu_i(a\to b)=\id_i=q(a\to b),&\text{if }a,b\in\Ob(\cA_i),\ i=0,1\\
0\to 1=q(a\to b),&\text{if }a\in\Ob(\cA_0)\text{ and }b\in\Ob(\cA_1).
\end{array}
\right.
\]
Suppose that there is another $u'\colon\cA\to\cC*\cD$ such that $pu'=q$ and that $u'$ restricts to $u_i$ on $\cA_i$. Then $u$ and $u'$ agree on $\cA_i$, and for any $a\to b$ in $\cA$ with $a\in\Ob(\cA_0)$, $b\in\Ob(\cA_1)$, $u'(a\to b)=u(a\to b)-0$ is the only morphism between $u(a)=u'(a)\in\Ob(\cC)$ and $u(b)=u'(b)\in\Ob(\cD)$. Hence $u=u'$.\\
\\
(2) Recall that $N(\cC)*N(\cD)$ is given by
\[
(N(\cC)*N(\cD))_n=\coprod_{\substack{i+1+j=n\\-1\leqslant i,j\leqslant n}}N(\cC)_i\times N(\cD)_j
\]
for each $[n]\in\Ob(\Delta)$. We then define a map
\[
\f_n\colon (N(\cC)*N(\cD))_n\to N(\cC*\cD)_n
\]
as below. Take an arbitrary $(x,y)\in N(\cC)_i\times N(\cD)_j$ with $-1\leqslant i,j\leqslant n$ and $i+1+j=n$, where $x$ or $y$ may be empty. Then $(x,y)$ corresponds to a unique $([i]\overset{u_0}{\to}\cC,[j]\overset{u_1}{\to}\cD)$ via the adjunction $\tau\dashv N$ plus the facts that the counit is an isomorphism and $\Delta^i=N([i])$. Moreover, let us define a functor $q\colon[n]\to[1]$ by sending $i\mapsto0$ and $i+1\mapsto 1$. Then by (1), we get a unique functor $u\colon[n]\to\cC*\cD$ such that $q=pu$ and $u|_{[i]}=u_0$, $u|_{[j]}=u_1$, where $p\colon\cC*\cD\to[1]$ is the same as in (1). Again under the adjunction, $u$ corresponds uniquely to a simplicial map $\Delta^n\to N(\cC*\cD)$ (a.k.a an element of $N(\cC*\cD)_n$), which we denote by $\f_n(x,y)$.

We claim that $\f_n$ is a bijection. To this end, we construct an inverse $\psi_n$ to $\f_n$. Take an element $z$ in $N(\cC*\cD)_n$, and it corresponds via adjunction to some $u\colon[n]\to\cC*\cD$. Put $q:=pu$, $i:=\max\{i\mid q(i)=0\}$ and $j:=n-i-1$. Then we can define $u_0\colon[i]\to \cC$ by restricting $u$ to $[i]$, and $u_1\colon[j]\to\cD$ by the composition $[j]\overset{k\mapsto k+i+1}{\longrightarrow}[n]\overset{u}{\to}\cC*\cD$, which actually lands in $\cD$. Again the pair $(u_0,u_1)$ corresponds under adjunction to an element of $N(\cC)_i\times N(\cD)_j$, for which we write $\psi_n(z)$. 

The well-definedness of $\f_n$ and $\psi_n$ lies in the adjunction bijection and the universal property of joins, which in every step of out construction provides a unique choice. 

Verifying $\psi_n$ and $\f_n$ being mutually inverse is straightforward. For example, to check that $\f_n\psi_n=\id_{N(\cC*\cD)_n}$, we consider an arbitrary $u\colon[n]\to\cC*\cD$ and $u_0,u_1$ constructed as above. By the universal property of joins, the $u'\colon[n]\to\cC*\cD$ such that $q=pu'$ and $u'|_{[i]}=u_0$, $u'|_{[j]}=u_1$ is unique (and thus equals to $u$), which corresponds to the image under $\f_n\psi_n$. The argument for $\psi_n\f_n=\id_{(N(\cC)*N(\cD))_n}$ is similar.

In what follows we show that the bijection $\f_n\colon (N(\cC)*N(\cD))_n\to N(\cC*\cD)_n$ is functorial in $[n]$. For this, we take a functor $f\colon[m]\to[n]$ and $-1\leqslant i,j\leqslant n$ such that $i+j+1=n$. Then there exists a unique pair of integers $(a,b)$  and functors $f_a\colon[a]\to[i]$, $f_b\colon[b]\to[j]$ satisfying $a+1+b=m$ and $f_a*f_b=f$. Explicitly, one has $a=\max\{a\mid f(a)\leqslant i\}$, $f_a(k)=f(k)$ and $f_b(k)=f(k+a+1)-i-1$. Consider the following diagram
\[
\begin{tikzcd}
(N(\cC)*N(\cD))_n\arrow[r]\arrow[d]& N(\cC*\cD)_n\arrow[d]\\
(N(\cC)*N(\cD))_m\arrow[r]& N(\cC*\cD)_m
\end{tikzcd}
\quad
\begin{tikzcd}[sep=small]
(x,y)\arrow[rr,mapsto]\arrow[d,mapsto]&& \f_n(x,y)\arrow[d,mapsto]\\
(f^*_ax,f^*_by)\arrow[r,mapsto]& \f_m(f^*_ax,f^*_by)& f^*\f_n(x,y)
\end{tikzcd}
\]
Note that under the adjunction, $f^*\f_n(x,y)$ corresponds to $u\circ f$, whereas $f^*_ax$, $f^*_by$ corresponds to $u_0\circ f_a$ and $u_1\circ f_b$, which correspond to some $u'\colon[m]\to\cC*\cD$. Note that the restriction of $u\circ f$ on $[a]$ and $[b]$ are respectively $u_0\circ f_a$ and $u_1\circ f_b$, and also that $p\circ u'=q_m=q_n\circ f=p\circ u\circ f$, where $q_m\colon[m]\to[1]$ and $q_n\colon[n]\to[1]$ are given by $[a]\mapsto0,[a+1]\mapsto1$ and $[i]\mapsto0,[i+1]\mapsto1$ respectively. By the universal property of joins (1), one has $u'=u\circ f$. Therefore $\f_m(f^*_ax,f^*_bx)=f^*\f_n(x,y)$, and in conclusion, $\f_n$ is functorial with regard to $[n]$.

So far we have proved $N(\cC*\cD)\cong N(\cC)* N(\cD)$.
\end{proof}

\newpage
\exercise{2}
\begin{proof}
    (1) Notice that $N(0)=\Delta^{-1}$. Now, applying (1.2), we see that
    $\Delta^i*\Delta^{n-i-1}=N([i])*N([n-i-1])\cong N([i]*[n-i-1])$, so it is
    enough to check that $[n]\cong [i]*[n-i-1]$.

    In $[i]*[n-i-1]$ there is exactly one morphism between any pair of objects
    coming from $[i]$ or from $[n-i-1]$. Also, given an object in $[i]$ and one
    in $[n-i-1]$, by definition of $[i]*[n-i-1]$ there is exactly one morphism
    between the two of them in this category, from the former to the latter.
    This shows that $[i]*[n-i-1]$ is an order and, since its set of objects has
    cardinality $n+1=(i+1)+((n-i-1)+1)$ like the one of $[n]$, we get that the
    two categories are (uniquely) isomorphic, as desired. \\

    (2) \\

    (3) Let's apply the operator $(-)^{\op}$ to the commutative diagram
    \[\begin{tikzcd}
	{\Lambda^n_k} & {X*Y} \\
	{\Delta^n} & {\Delta^1}
	\arrow["{u}", from=1-1, to=1-2]
	\arrow["{v}"', from=2-1, to=2-2]
	\arrow[from=1-1, to=2-1]
	\arrow["{p}", from=1-2, to=2-2]
    \end{tikzcd},\]
    giving us a commutative diagram which admits a filler $g$ by (2.2). Here we
    use the fact that $(X*Y)^{\op}\cong Y^{\op}*X^{\op}$.
    \[\begin{tikzcd}
	{\Lambda^n_{n-k}} & {Y^{\op}*X^{\op}} \\
	{\Delta^n} & {\Delta^1}
	\arrow["{u^{\op}}", from=1-1, to=1-2]
	\arrow["{v^{\op}}"', from=2-1, to=2-2]
	\arrow[from=1-1, to=2-1]
	\arrow["{p^{\op}}", from=1-2, to=2-2]
	\arrow["{g}" description, from=2-1, to=1-2, dotted]
    \end{tikzcd}\]
    By reapplying the operator (which is an involution) we get then the desired
    filler $f=g^{\op}$.
    \[\begin{tikzcd}
	{\Lambda^n_k} & {X*Y} \\
	{\Delta^n} & {\Delta^1}
	\arrow["{u}", from=1-1, to=1-2]
	\arrow["{v}"', from=2-1, to=2-2]
	\arrow[from=1-1, to=2-1]
	\arrow["{p}", from=1-2, to=2-2]
	\arrow["{f}" description, from=2-1, to=1-2, dotted]
    \end{tikzcd}\] \\

    (4) Since the diagram is commutative and the map on the left is a
    monomorphism bijective on objects, the fact that $v(j)=0$ is equivalent
    to $pu(j)=0$ and therefore, by definition of $p$ and $i$, $u(j)\in X_0$ for
    all $0\leq j\leq i$, $u(j)\in Y_0$ for all $i<j\leq n$.

    Suppose to have a lifting $f$ already. We will start showing its uniqueness
    by rewriting $\Delta^n$ as $\Delta^i*\Delta^{n-i-1}$. This gives us the
    restrictions $v|_{\Delta^i}=v|_{v^{-1}(0)}$,
    $v|_{\Delta^{n-i-1}}=v|_{v^{-1}(1)}$, which map all 0-simplices respectively
    to 0 and 1 by our previous ovservation. Precomposing by the inclusion
    $\Lambda^n_i\rightarrow\Delta^n_i$, we get that $v|_{\Delta^i}=pu|_{\Delta^i}$,
    $v|_{\Delta^{n-i-1}}=pu|_{\Delta^{n-i-1}}$, thus all of $\Delta^i$ is sent
    to $X$ and all of $\Delta^{n-i-1}$ to $Y$ under $u$ by the description of
    $p$. This allows us to construct the following commutative diagram

    \[\begin{tikzcd}
	&& {X\sqcup Y} \\
	{\Delta^i\sqcup\Delta^{n-i-1}} & {\Lambda^n_i} & {X*Y} \\
	{\Delta^i*\Delta^{n-i-1}} & {\Delta^n} & {\Delta^1} \\
	{\partial\Delta^1} \\
	\arrow["{u}", from=2-2, to=2-3]
	\arrow["{v}", from=3-2, to=3-3]
	\arrow[from=2-2, to=3-2, hook]
	\arrow["{p}", from=2-3, to=3-3]
	\arrow[from=3-1, to=3-2, shift right=1, no head]
	\arrow[from=3-1, to=3-2, no head]
	\arrow[from=2-1, to=3-1, hook]
	\arrow[from=1-3, to=2-3, hook]
	\arrow[from=2-1, to=2-2, hook]
    \arrow["{f}" description, from=3-2, to=2-3, dotted]
	\arrow["{u|_{\Delta^i}\sqcup u|_{\Delta^{n-i-1}}}", from=2-1, to=1-3]
	\arrow[from=2-1, to=4-1, curve={height=46pt}]
	\arrow[from=4-1, to=3-3]
    \end{tikzcd}\]
    Now, restricting our focus to the commutative diagram
    \[\begin{tikzcd}
	& {X\sqcup Y} & {X*Y} \\
	{\Delta^i\sqcup\Delta^{n-i-1}} & {\Delta^n} \\
	{\partial\Delta^1} & {\Delta^1} \\
	\arrow[from=1-2, to=1-3, hook]
	\arrow["{v}", from=2-2, to=3-2]
	\arrow["{p}", from=1-3, to=3-2, curve={height=-6pt}]
	\arrow["{f}" description, from=2-2, to=1-3, dotted]
	\arrow["{u|_{\Delta^i}\sqcup u|_{\Delta^{n-i-1}}}", from=2-1, to=1-2]
	\arrow[from=2-1, to=3-1]
	\arrow[from=3-1, to=3-2, hook]
	\arrow[from=1-2, to=3-1, curve={height=-6pt}]
	\arrow[from=2-1, to=2-2, hook, crossing over]
    \end{tikzcd},\]
    we see that there can be at most one $f$ solving our initial lifting problem
    since all solutions must fill this diagram and we have the universal
    property of the join.

    Notice now that
    $u|_{\Delta^i}*u|_{\Delta^{n-i-1}}\colon\Delta^n\cong\Delta^i*\Delta^{n-i-1}\rightarrow
    X*Y$ solves the lifting problem we started from by construction, hence the
    thesis. \\

    (5) It is enough to check that $ho(X*Y)$ has the universal property of the
    join of $ho(X)$ and $ho(Y)$. Let's consider then functors
    $q\colon\cA\rightarrow [1]$, $u_0\colon\cA_0\rightarrow ho(X)$,
    $u_1\colon\cA_1\rightarrow ho(Y)$, and the obvious embedding $ho(X)\sqcup
    ho(Y)\rightarrow ho(X*Y)$ (it's
    faithful because joining two $\infty$-categories does not produce new
    2-simplices establishing homotopy relations between the morphisms in $X$ or
    in $Y$). $p\colon ho(X*Y)\rightarrow [1]$ will be given by $x\mapsto 0$,
    $y\mapsto 1$, and $p(x\rightarrow y)=(0\rightarrow 1)$. Notice that there's
    no map $y\rightarrow x$ since it would come from $(y,x)\in Y_0\times X_0$.
    \[\begin{tikzcd}
	& {ho(X)\sqcup ho(Y)} & {ho(X*Y)} \\
	{\cA_0\sqcup\cA_1} & {\cA} \\
	{[0]\sqcup[0]} & {[1]}
	\arrow["{u_0\sqcup u_1}", from=2-1, to=1-2]
	\arrow["{f}" description, from=2-2, to=1-3, dotted]
	\arrow[from=1-2, to=1-3, hook]
	\arrow["{q}", from=2-2, to=3-2]
	\arrow["{p}", from=1-3, to=3-2, curve={height=-6pt}]
	\arrow[from=2-1, to=3-1]
	\arrow[from=3-1, to=3-2, hook]
	\arrow[from=1-2, to=3-1, curve={height=-6pt}]
	\arrow[from=2-1, to=2-2, hook, crossing over]
    \end{tikzcd}\]
    To construct a factorization $f\colon\cA\rightarrow ho(X*Y)$ of $q$ making
    the diagram commute we are forced to start
    by composing $u_0\sqcup u_1$ with the embedding, which gives us
    $a_i\mapsto u_i(a)$ for $a_i\in\Ob(\cA_i)$, $g\mapsto u_i(g)$ for
    $g\in\mor(\cA_i)$. To extend then this functor to $\cA$, we have to to send
    maps $a_0\rightarrow a_1$ to the unique morphism $f(a_0)\rightarrow f(a_1)$
    given by the element $(f(a_0),f(a_1))\in X_0\times Y_0\subset (X*Y)_1$.
    Notice that there are no morphisms $a_1\rightarrow a_0$ in $\cA$ by the
    definition of the $\cA_i$ since they would need to be mapped to an arrow
    $1\rightarrow 0$ under $q$, but it is not there.

    We see that identities are trivially preserved and compositions of arrows all in $\cA_i$
    are too since the $u_i$ and the embedding are functors. If one composes
    instead an arrow $a'_0\rightarrow a_0$ with one $a_0\rightarrow a_1$ whose
    domain and codomain lie in different
    categories the result is again a map $a'_0\rightarrow a_1$ with domain and
    codomain lying in
    different categories and will therefore be mapped to the unique map
    $f(a'_0)\rightarrow f(a_1)$. Likewise, the composition of the maps one
    obtains by first applying $f$ and then composing in $ho(X*Y)$ is again the
    unique map $f(a'_0)\rightarrow f(a_1)$. A symmetric argument for
    $a_0\rightarrow a_1$ and $a_1\rightarrow a'_1$ then gives us functoriality.

    By construction, the desired diagram commutes and uniqueness of
    factorization follows from the fact that when we were defining $f$
    we had a unique possible choice at every step.
\end{proof}

\end{document}
